\documentclass[twocolumn, 10pt]{article}
\usepackage[a4paper, margin=0pt]{geometry}
\usepackage{enumitem}
\usepackage{amsmath}
\usepackage{graphicx}
\usepackage{xcolor}
\usepackage{tabularray}
\usepackage{mathtools}

\setlength{\parindent}{0pt}

\begin{document}

\textbf{ArchiMate} \\[2pt]
\textbf{Business Service} - conceito de processo de negócio (“business process”) pode realizar

\textbf{Composition} - relação que modela que um application component tem application interface que é exclusiva desse application component

\textbf{Flow} - tipo de relação entre elementos de comportamento que representa troca de informação entre processos ou funções
\\[6pt]
\textbf{BPMN} \\[2pt]
\textbf{Data store} - 

\textbf{Non-interrupting Event} - 

\textbf{Ad-hoc Marker} - identificar que a atividade A é composta pelas tarefas A1, A2 e A3, sem ser relevante a explicitação de qualquer ordem entre elas

\end{document}