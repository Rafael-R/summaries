\documentclass[12pt]{article}
\usepackage[paper=letterpaper,margin=2cm]{geometry}
\usepackage{enumitem}
\usepackage{amsmath}
\usepackage{graphicx}
\usepackage{xcolor}
\usepackage{tabularray}
\usepackage{mathtools}

\graphicspath{ {./images/} }
\setlength{\parindent}{0pt}

\begin{document}

\begin{titlepage}
    \begin{center}
        \vspace*{1cm}

        \textbf{Análise e Modelação de Sistemas}
        \vspace{0.5cm}

        Resumo
        \vspace{1.5cm}

        \textbf{Rafael Rodrigues}
        \vfill
        LEIC \\
        Instituto Superior Técnico \\
        2022/2023
    \end{center}
\end{titlepage}

\tableofcontents

\newpage

\section{Introduction}

\textbf{Engenharia de Sistemas} - disciplina que considera diversos fatores complexos de forma a criar e implementar sistemas úteis.
\\[6pt]
\textbf{Hard Systems} - problema bem definido cujo ponto de vista é partilhado pelos "stakeholders" envolvidos, permitindo uma abordagem científica na resolução do problema. Lida apenas com fatores ligados à engenharia e o seu propósito é definir uma solução ideal que lida com todos os problemas dos "stakeholders". 
\\[6pt]
\textbf{Soft Systems} - problema vago ligado a um objetivo pouco preciso onde os "stakeholders" envolvidos têm uma interpretação diferente do problema permitindo diferentes abordagens de resolução do problema distintas. Lida com fatores sócio-técnicos e humanos para além dos de engenharia. Encontrar uma solução ideal pode não ser um objetivo, propósitos alternativos incluem aprendizagem e melhor compreensão do sistema.
\\[6pt]
\textbf{Black-Box} \\
Perspetiva do utilizador (qual é a funcionalidade?)
\begin{itemize}[topsep=0pt, itemsep=0pt]
    \item Funcionalidade: input e output
    \item Comportamento: manifestação da funcionalidade com o passar do tempo
\end{itemize} 
\vspace{6pt}
\textbf{White-Box} \\ 
Perspetiva do construtor (como é feito?)
\begin{itemize}[topsep=0pt, itemsep=0pt]
    \item Construção: componentes e relações de interação
    \item Operação: manifestação dos componentes com o passar do tempo
\end{itemize}
\vspace{6pt}
\textbf{Sistema} - conjunto de entidades que interagem com o objetivo específico que não pode ser realizado por nenhuma das entidades individuais.
\\[6pt]
\textbf{Abordagem de Engenharia de Sistemas}
\begin{itemize}[topsep=0pt, itemsep=0pt]
    \item Identificação e quantificação dos objetivos
    \item Criação de design alternativos
    \item Seleção e implementação do melhor design
    \item Verificação do design
    \item Verificação pós-implementação
\end{itemize}
\vspace{6pt}
\textbf{Sistema de Informação} - sistema de processamento de informação em conjunto com recursos organizacionais humanos, técnicos e financeiros que providenciam e distribuem informação.
\\[6pt]
\textbf{Modelo} - artefacto que representa algo.
\\[6pt]
\textbf{Arquitetura} - define os concelhos ou prioridades de um sistema num ambiente através dos seus elementos, relações e princípios do seu design e evolução.

\newpage

\section{Engenharia de Requisitos}

\textbf{Requisito}:
\begin{itemize}[topsep=0pt, itemsep=0pt]
    \item Condição ou capacidade necessária para que o utilizador resolva um problema ou atinja um objetivo.
    \item Condição ou capacidade que tem de ser cumprida ou possuída por um sistema ou componente para satisfazer um contrato, standard, especificação ou outra formalidade imposta por documentação.
\end{itemize}
\vspace{6pt}
\textbf{Functional Requirements (FR)} - declarações de serviços que o sistema deve proporcionar, como deve reagir a inputs específicos e situações. Em alguns casos o requisito deve referir o que o sistema não deve fazer.
\\[6pt]
\textbf{Non-Functional Requirements (NFR)} - ambíguo e pouco específico.
\\[6pt]
\textbf{Requisito de Usabilidade} - são definidos de forma mais vaga e abstrata do que os requisitos funcionais.
\\[6pt]
\textbf{Requisito de Qualidade} - define uma propriedade ou qualidade do sistema ou de um dado componente, serviço ou função.
\\[6pt]
\textbf{Restrição} - é um requisito organizacional ou tecnológico que restringe a forma como o sistema é desenvolvido. Existem restrições que afetam o sistema ou o desenvolvimento.
\\[6pt]
\textbf{User Requirement (Use case)} - representa um conjunto de ações que o utilizador realiza no sistema de modo a atingir um resultado.
\\[6pt]
\textbf{Business Requirement (Goal)} - especificação da intenção relacionada com os objetivos, propriedades ou uso do sistema.
\\[6pt]
\textbf{Cenário} - exemplo concreto de satisfação ou falha na satisfação de um goal, definido o conjunto de passos de interação executados.

\subsubsection*{SMART Requirements}

\begin{itemize}
    \item \textbf{Specific} - claro, consistente, simples, apropriadamente detalhado.
    \item \textbf{Measurable} - quantificável e avaliável quanto ao cumprimento de objetivos.
    \item \textbf{Achievable} - possível exibir um comportamento de acordo com os requisitos.
    \item \textbf{Realizable} - possível de ser feito de acordo com as restrições de desenvolvimento.
    \item \textbf{Traceable} - possível seguir um requisito dada a sua especificação até ao design, implementação e teste.  
\end{itemize}

\textbf{Requirements elicitation} - é uma das 3 atividades básicas da engenharia de requisitos e o seu objetivo é:
\begin{enumerate}[topsep=0pt, itemsep=0pt]
    \item Identificar fontes importantes de requisitos.
    \item Obter requisitos das fontes identificadas.
    \item Desenvolver requisitos novos e inovadores.
\end{enumerate}

\textbf{Requirements Documentation} - documentação e especificação dos requisitos "elicited" de acordo com a documentação definida e regras especificadas.
\\[6pt]
\textbf{Requirements Negotiation} - explicitar conflitos entre os pontos de vista dos "stakeholders" e resolvê-los.
\\[6pt]
\textbf{Requirements Validation} - validação de artefactos requeridos, atividades chave e contexto.
\\[6pt]
\textbf{Requirements Traceability} - é possível seguir o desenvolvimento de um requisito em qualquer fase do processo.

\newpage

\section{ArchiMate}



\newpage

\section{BPMN}

\textbf{Activity} - a unit of work.
\\[6pt]
\textbf{Event} - And occurrence during a business process.
\\[6pt]
\textbf{Gateway} - controls the flow of activities.


\newpage

\section{UML}

\end{document}