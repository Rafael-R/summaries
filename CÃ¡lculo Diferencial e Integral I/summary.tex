\documentclass[11pt]{article}
\usepackage[paper=letterpaper,margin=2cm]{geometry}
\usepackage{enumitem}
\usepackage{amsmath}
\usepackage{amsfonts}
\usepackage{tabularray}
\usepackage{mathtools}

\setlength{\parindent}{0pt}

\newcommand{\twopartdef}[4]
{ \displaystyle
	\left\{
		\begin{array}{ll}
			#1, & \mbox{se } #2 \\
			#3, & \mbox{se } #4
		\end{array}
	\right.
}

\begin{document}

\begin{titlepage}
    \begin{center}
        \vspace*{1cm}

        \textbf{\LARGE Cálculo Diferencial e Integral I}
        \vspace{0.5cm}

        \Large Resumo
        \vspace{1.5cm}

        \textbf{Rafael Rodrigues}
        \vfill
        LEIC \\
        Instituto Superior Técnico \\
        2023/2024
    \end{center}
\end{titlepage}

\tableofcontents

\newpage

\section{\MakeUppercase{Números Reais: \\ Revisões e Propriedades}}

\subsection{Propriedades dos Números Reais}

\begin{itemize}[topsep=0pt]
    \item Números Naturais: $\mathbb{N} = \left\{1, 2, 3, ...\right\}$
    \item Números Inteiros: $\mathbb{Z} = \left\{..., -2, -1, 0, 1, 2, ...\right\}$
    \item Números Racionais: $\mathbb{Q} = \displaystyle \left\{\frac{p}{q}:p,q\in \mathbb{Z},\ q \neq 0\right\}$
\end{itemize}

\subsubsection{Propriedades Básicas das Operações}

\begin{enumerate}
    \item Propriedades da Soma:
    \begin{itemize}[topsep=0pt]
        \item Propriedade Comutativa: $a + b = b + a$
        \item Propriedade Associativa: $\left(a + b\right) + c = a + \left(b + c\right)$
        \item Elemento Neutro: $a + 0 = 0 + a = a$
        \item Elemento Simétrico: $a + \left(-a\right) = \left(-a\right) + a = 0$
    \end{itemize}
    \item Propriedades da Produto
    \begin{itemize}[topsep=0pt]
        \item Propriedade Comutativa: $a \cdot b = b \cdot a$
        \item Propriedade Associativa: 
            $\left(a \cdot b\right) \cdot c = a \cdot \left(b \cdot c\right)$
        \item Elemento Neutro: $a \cdot 1 = 1 \cdot a = a$
        \item Elemento Inverso: 
            $\displaystyle a \cdot \frac{1}{a} = \frac{1}{a} \cdot a = 1 \ 
            \left(\frac{1}{a} = a^{-1}\right)$
        \item Propriedade Distributiva: $(a + b) \cdot c = a \cdot c + b \cdot c$
    \end{itemize}
    \item Propriedades da Relação (Ordem)
    \begin{itemize}[topsep=0pt]
        \item Relação de ordem: $?$
        \item Compatibilidade da Soma: $a \leq b \Rightarrow a + c \leq b + c$
        \item Compatibilidade do Produto: 
            $a \leq b \ e \ c > 0 \ ent\tilde{a}o \ ac \leq bc$
    \end{itemize}
\end{enumerate}

\subsubsection{Majorantes e Minorantes}

Seja $A \subset \mathbb{R}$:
\begin{itemize}[topsep=2pt, itemsep=2pt]
    \item $A$ diz-se $majorado$ se existe $b \in \mathbb{R}$ tal que $x \leq b$, para qualquer $x \in A$ (ou seja, $A \subseteq\ ]-\infty ,b]$). \\ Neste caso, b diz-se um majorante de A.
    \item $A$ diz-se $minorado$ se existe $a \in \mathbb{R}$ tal que $x \geq a$, para qualquer $x \in A$ (ou seja, $A \subseteq\ [a, +\infty[$). \\ Neste caso, a diz-se um minorante de A.
    \item $A$ diz-se $limitado$ se é majorado e minorado. Neste caso, existem $a, b \in \mathbb{R}$ tais que $a \leq x \leq b$, para qualquer $x \in A$ (ou seja $A \subset [a, b]$).
\end{itemize}

\subsubsection{Supremo e Ínfimo}

Seja $A \subset \mathbb{R}$:
\begin{itemize}[topsep=2pt, itemsep=2pt]
    \item Chama-se \textit{supremo} de $A\ (\sup A)$ ao menor dos majorantes de $A$, se existir.
    \item Chama-se \textit{ínfimo} de $A\ (\inf A)$ ao maior dos minorantes de $A$, se existir.
\end{itemize}

\subsubsection{Máximo e Mínimo}

Definimos o \textit{máximo} e \textit{mínimo} de um conjunto como o maior e o menor dos seus elementos (se existirem). Isto é equivalente a dizer que:
\begin{itemize}[topsep=2pt, itemsep=2pt]
    \item $\max A = M$ se $M$ é majorante e $M \in A$;
    \item $\min A = m$ se $m$ é minorante e $m \in A$.
\end{itemize}

\subsection{Módulo}

O módulo ou valor absoluto de um número real $x \in R$ é definido por:
\begin{align*}
    |x| = \twopartdef{x}{x \geq 0}{-x}{x < 0}
\end{align*}

\subsubsection{Propriedades do Módulo}

\begin{itemize}[itemsep=2pt]
    \item $|x| \geq 0$
    \item $|x| = 0 \Leftrightarrow  x = 0$
    \item $|-x| = |x|$
    \item $|xy| = |x||y|$
    \item $|x|^2 = \left\lvert x^2\right\rvert = x^2$
    \item $|x + y| \leq |x| + |y|$
    \item $x^2 < a^2 \Leftrightarrow |x| < |a|$
\end{itemize}

\subsubsection{Inequações com Módulos}

\begin{align*}
    & \left\lvert x \right\rvert < b\ \Leftrightarrow\ x < b\,\land\,x > -b\  \Leftrightarrow\ -b < x < b\ \\
    & \left\lvert x \right\rvert > b\ \Leftrightarrow\ x > b\,\lor \,x < -b
\end{align*}

\subsubsection{Vizinhança}

Define-se a \textit{vizinhança} de $centro\ a$ e $raio\ R > 0$, 
\begin{align*}
    V_R(a) = \left\{x \in \mathbb{R}:|x-a| < R\right\} =\ ]a-R, a+R[
\end{align*}
como o conjunto dos pontos cuja distância a $a$ é inferior a $R$.

\subsection{Método de Indução}

Seja $P(n)$ uma proposição para cada $n\in\mathbb{N}$. Suponhamos que:
\begin{enumerate}
    \item $P(1)$ é verdadeira;
    \item sempre que $P(n)$ é verdadeira para $algum\ n$, então $P(n+1)$ também é verdadeira.
\end{enumerate}
Então conclui-se que $P(n)$ é verdadeira para $todo$ o $n\in\mathbb{N}$.

\subsubsection{Demonstração}

\begin{enumerate}
    \item Verificar que $P(1)$ é verdadeiro. (\textit{base de indução})
    \item Assumindo que $P(n)$ é verdadeiro (\textit{hipótese de indução}) provar que $P(n+1)$ é verdadeiro. (\textit{tese})
\end{enumerate}

\subsection{Somatórios}

\subsubsection{Propriedades do Somatório}

\begin{itemize}
    \item Propriedade aditiva: $\displaystyle \sum_{k=1}^{n}\left(a_k+b_k\right) = \sum_{k=1}^{n}a_k + \sum_{k=1}^{n}b_k$
    \item Homogeneidade: $\displaystyle \sum_{k=1}^{n}(ca_k) = c \sum_{k=1}^{n}a_k$ para qualquer constante $c\in\mathbb{R}$
    \item Propriedade telescópica: $\displaystyle \sum_{k=1}^{n}(a_k-a_{k+1}) = a_1-a_{n+1}$
    \item $\displaystyle \sum_{k=1}^{n}a_k = \sum_{k=p+1}^{p+n}a_{k-p}$ para qualquer $p\in\mathbb{N}$
\end{itemize}

\newpage

\section{\MakeUppercase{Funções Reais de Variável Real: \\ Limite e Continuidade}}



\subsection{Classes de funções elementares}

\begin{itemize}
    \item Função Polinomial
    \item Função Racional
    \item Função Exponencial
    \item Funções Trigonométricas:
        \begin{itemize}[topsep=0pt]
            \item Seno:
            \item Cosseno:
            \item Tangente:
            \item Cotangente:
        \end{itemize}
    \item Funções Hiperbólicas:
        \begin{itemize}[topsep=0pt]
            \item Seno:
            \item Cosseno:
        \end{itemize}
    \item Função de Heaviside
    \item Função de Dirichlet
    \item Função Composta
\end{itemize}

\subsection{Funções Injetivas e suas Inversas}

Uma função $f:D\subset\mathbb{R}\rightarrow\mathbb{R}$ diz-se $injetiva$ se, para qualquer valor do contradomínio $y \in f(D)$, existir 

\subsection{Limite de uma função num ponto}

\subsection{Propriedades do Limite de Funções num Ponto}

\subsection{Limites Relativos e Laterais}

\subsection{Limites na recta acabada}

\subsection{Continuidade de Funções Reais de Variável Real}

\subsection{Algumas Propriedades Locais das Funções Contínuas}

\subsection{Propriedades Globais das Funções Contínuas}



\newpage

\section{\MakeUppercase{Cálculo Diferencial}}



\newpage

\section{\MakeUppercase{Primitivação}}

\subsection{Definição de Primitiva e Aplicações}

\subsection{Primitivas Imediatas}

\subsection{Primitivas Quase-Imediatas}

\subsection{Primitivação por Partes}

\subsection{Primitivas de Funções Racionais}

\subsection{Primitivação por Substituição}

\subsection{Primitivação de Funções Polinomiais de Senos e Cossenos}

\subsection{Primitivação de Funções Racionais de Senos e Cossenos}

\subsection{Resolução de equações diferenciais ordinárias de variáveis separáveis}



\newpage

\section{\MakeUppercase{Cálculo Integral}}

\subsection{Motivação para a Noção de Integral}

\subsection{Partições, Somas Inferiores e Superiores}

\subsection{Integral Superior e Inferior}

\subsection{Funções Integráveis e Nao-Integráveis}

\subsection{Critérios de integrabilidade e exemplos}

\subsection{Propriedades do integral}

\subsection{Integral Indefinido}

\subsection{Teorema Fundamental do Calculo. Regra de Barrow}

\subsection{Algumas aplicações do integral}



\newpage

\section{\MakeUppercase{Sucessões e Séries}}

\subsection{\MakeUppercase{Sucessões}}

\subsubsection{Introdução e exemplos}

\subsubsection{Limite de uma Sucessão}

\subsubsection{Propriedades do Limite de Sucessões}

\subsubsection{Sucessões Monótonas e Limitadas}

\subsubsection{Escala de Sucessões}

\subsection{\MakeUppercase{Séries Numéricas}}

\subsubsection{Introdução e exemplos}

\subsubsection{Operações algébricas sobre séries}

\subsubsection{Condição necessária para convergência}

\subsubsection{Séries de Termos Não-Negativos e Critérios de Convergência}

\subsubsection{Convergência Simples e Absoluta}

\subsubsection{Séries Alternadas e Critérios de Convergência}

\subsection{\MakeUppercase{Séries de Potências}}

\subsubsection{Definições e resultados gerais}

\subsubsection{Séries de Taylor}



\end{document}