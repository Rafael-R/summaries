\documentclass[11pt, a4paper]{article}
\usepackage[margin=2cm]{geometry}
\usepackage{hyperref}
\usepackage{enumitem}
\usepackage{amsmath,amsfonts,amssymb, amstext}
\usepackage{tabularray}
\usepackage{array}
\usepackage{cancel}

\newcolumntype{C}{>{$}c<{$}}

\hypersetup{
    colorlinks,
    citecolor=black,
    filecolor=black,
    linkcolor=black,
    urlcolor=black
}

\setlength{\parindent}{0pt}

\begin{document}

\begin{titlepage}
    \begin{center}
        \vspace*{5cm}

        \textbf{\LARGE Cálculo Diferencial e Integral III}
        \vspace{1cm}

        \Large Resumo
        \vspace{2cm}

        \textbf{Rafael Rodrigues}
        \vfill
        LEIC \\
        Instituto Superior Técnico \\
        2023/2024
    \end{center}
\end{titlepage}

\tableofcontents

\newpage


\section{Equações Diferenciais Ordinárias}

\subsection{Equações Escalares de 1ª Ordem}

\subsubsection{Equações Lineares}

\begin{equation*}
    \frac{dy}{dt} + a(t)y = b(t)
\end{equation*}

\begin{enumerate}
    \item Verificar que está escrita na forma linear
    \item Determinar o fator integrante: $\displaystyle \mu(t) = \exp\left(\int a(t) \, dt \right)$
    \item Determinar $y$:
          $\displaystyle y(t) = \frac{1}{\mu(t)}\int \mu(t)b(t) \, dt + C$
\end{enumerate}

\subsubsection{Equações Separáveis}

\begin{equation*}
    f(y)\frac{dy}{dt} = g(t) \Leftrightarrow
    f(y) \, dy = g(t) \, dt
\end{equation*}

\begin{enumerate}
    \item Verificar que está escrita na forma separável
    \item Integrar ambos os membros da igualdade:
          $\displaystyle \int f(y) \, dy = \int g(t) \, dt + C$
\end{enumerate}

\subsubsection{Equações Exatas}

\begin{equation*}
    M(t,y) + N(t, y)\frac{dy}{dt} = 0 \Leftrightarrow
    M(t,y) \, dt + N(t, y) \, dy = 0
\end{equation*}

\begin{enumerate}
    \item Verificar se a equação é exata:
          $\displaystyle \frac{\partial M}{\partial y} =
              \frac{\partial N}{\partial t} \Rightarrow
              \exists\ \Phi: \mathbb{R}^2 \rightarrow \mathbb{R}$ tal que
          $\nabla \Phi(t, y) = (M, N)$
    \item Encontrar um potencial para $\Phi(t, y)$:
          \begin{equation*}
              \begin{cases}
                  \Phi = \int M \,dx \\
                  \Phi = \int N \,dy
              \end{cases} \Rightarrow
              \Phi = C
          \end{equation*}
    \item Isolar o $y$
\end{enumerate}

\subsubsection{Equações Redutíveis a Exatas}

\begin{equation*}
    M(t,y) + N(t, y)\frac{dy}{dt} = 0
\end{equation*}

\begin{enumerate}
    \item Quando a equação não é exata:
          $\displaystyle \frac{\partial M}{\partial y} \neq
              \frac{\partial N}{\partial t}$
    \item Calcular as razões:
          \begin{equation*}
              \frac{M_y - N_t}{N} \text{\ \ \ \ e \ \ \ } \frac{N_t - M_y}{M}
          \end{equation*}
    \item Escolher a razão que depender apenas de uma variável e calcular o
          \textbf{fator integrante}:
          \begin{equation*}
              \mu = \exp\left(\int \text{razão escolhida}\right)
          \end{equation*}
    \item Resolver a \textbf{equação exata}:
          $\displaystyle \mu M(t,y) + \mu N(t, y)\frac{dy}{dt} = 0$

\end{enumerate}

\subsection{Equações Vetoriais Lineares}

\subsubsection*{Valores Próprios}

\begin{equation*}
    \text{det}(A - \lambda I) = 0 \Leftrightarrow 
    \left\lvert \begin{matrix} 
        a - \lambda_1 & b \\ c & d - \lambda_2
    \end{matrix}\right\rvert = 0 \Leftrightarrow
    (a - \lambda_1)(d - \lambda_2) - cd = 0
\end{equation*}

\subsubsection*{Vetores Próprios}

\begin{itemize}
    \item $\lambda_1 \ne \lambda_2$
        \begin{equation*}
            \begin{matrix}
                \left[A - \lambda_1 I\right] \cdot \vec{v}_1 = 0 & \text{e} &
                \left[A - \lambda_2 I\right] \cdot \vec{v}_2 = 0
            \end{matrix}
        \end{equation*}
    \item $\lambda_1 = \lambda_2$
        \begin{equation*}
            \begin{matrix}
                \left[A - \lambda I\right] \cdot \vec{v}_1 = 0 & \text{e} &
                \left[A - \lambda I\right] \cdot \vec{v}_2 = \vec{v}_1
            \end{matrix}
        \end{equation*}
    \item $\lambda = a \pm bi$
        \begin{equation*}
                \left[A - (a \pm bi)\, I\right] \cdot \vec{v} = 0
        \end{equation*}
\end{itemize}

\subsubsection{Exponencial de uma Matriz}

\begin{itemize}
    \item Diretas
    \begin{enumerate}
        \item Matriz Diagonal
        \begin{equation*}
            A = \left[\begin{matrix}
                \alpha & \cdots & 0 \\
                \vdots & \ddots & \vdots \\
                0 & \cdots & \beta 
            \end{matrix}\right] 
            \ \ \ \Rightarrow\ \ \ 
            e^{At} = \left[\begin{matrix}
                e^{\alpha t} & \cdots & 0 \\
                \vdots & \ddots & \vdots \\
                0 & \cdots & e^{\beta t} 
            \end{matrix}\right]
        \end{equation*}
        \item Bloco de Jordan
        \begin{equation*}
            A = \left[\begin{matrix}
                \alpha & 1 & \cdots & 0 \\
                0 & \ddots & \ddots & \vdots \\
                \vdots & \ddots & \ddots & 1 \\
                0 & \cdots & 0 & \alpha 
            \end{matrix}\right]
            \ \ \ \Rightarrow\ \ \ 
            e^{At} = \left[\begin{matrix}
                e^{\alpha t} & e^{\alpha t}t & \cdots & e^{\alpha t}\frac{t^n}{n!} \\
                0 & \ddots & \ddots & \vdots \\
                \vdots & \ddots & \ddots & e^{\alpha t}t \\
                0 & \cdots & 0 & e^{\alpha t}
            \end{matrix}\right]
        \end{equation*}
    \end{enumerate}
    \item Indiretas
    \begin{equation*}
        e^{At} = S \cdot e^{\Lambda t} \cdot S^{-1}
    \end{equation*}
    \begin{enumerate}
        \item Calcular os \textbf{valores próprios} de $A$:
        \begin{enumerate}
            \item Se $\lambda_1 \ne \lambda_2$:
            \begin{equation*}
                \Lambda = \left[\begin{matrix}
                    \lambda_1 & 0 \\
                    0 & \lambda_2
                \end{matrix}\right] \ \ \ \ \Rightarrow \ \ \ \ 
                e^{\Lambda t} = \left[\begin{matrix}
                    e^{\lambda_1 t} & 0 \\
                    0 & e^{\lambda_2 t}
                \end{matrix}\right]
            \end{equation*}
            \item Se $\lambda_1 = \lambda_2$:
            \begin{equation*}
                \Lambda = \left[\begin{matrix}
                    \lambda & 1 \\
                    0 & \lambda
                \end{matrix}\right] \ \ \ \ \Rightarrow \ \ \ \ 
                e^{\Lambda t} = \left[\begin{matrix}
                    e^{\lambda t} & e^{\lambda t}t \\
                    0 & e^{\lambda t}
                \end{matrix}\right]
            \end{equation*}
            \item Se $\lambda = a \pm bi$:
            \begin{equation*}
                \Lambda = \left[\begin{matrix}
                    a+bi & 0 \\
                    0 & a-bi
                \end{matrix}\right] \ \ \ \ \Rightarrow \ \ \ \ 
                e^{\Lambda t} = \left[\begin{matrix}
                    e^{(a+bi) t} & 0 \\
                    0 & e^{(a-bi) t}
                \end{matrix}\right]
            \end{equation*}
        \end{enumerate}
        \item Calcular os \textbf{vetores próprios} de $A$:
        \begin{equation*}
            S = \left[\begin{matrix}
                \vdots & \vdots \\
                v_1 & v_2 \\
                \vdots & \vdots
            \end{matrix}\right]
            \ \ \ \ \ \ \ \ \ \
            S^{-1} = \frac{1}{\text{det}} \left[\begin{matrix}
                d & -b \\
                -c & a 
            \end{matrix}\right]
        \end{equation*}
    \end{enumerate}
\end{itemize}

\subsubsection{Caso Homogéneo}

\begin{equation*}
    X' = AX \Leftrightarrow
    \left[ \begin{matrix} x' \\ y' \end{matrix} \right] = 
    \left[ \begin{matrix} a & b \\ c & d \end{matrix} \right]
    \left[ \begin{matrix} x \\ y \end{matrix} \right]
\end{equation*}

\begin{itemize}
    \item $\lambda_1 \ne \lambda_2$
        \begin{equation*}
            X(t) = c_1 e^{\lambda_1 t}[\vec{v}_1] + 
                c_2 e^{\lambda_2 t}[\vec{v}_2]
        \end{equation*}
    \item $\lambda_1 = \lambda_2$
        \begin{equation*}
            X(t) = c_1 e^{\lambda_1 t}[\vec{v}_1] + 
                c_2 e^{\lambda_1 t}[t\vec{v}_1 + \vec{v}_2]
        \end{equation*}
    \item $\lambda = a \pm bi$
        \begin{equation*}
            X(t) = c_1 [\text{Re}] + c_2 [\text{Im}]
        \end{equation*}
\end{itemize}

\subsubsection*{Solução com Matriz Exponencial}

\begin{equation*}
    X(t) = e^{A(t-t_0)} \cdot X(t_0)
\end{equation*}

\subsubsection{Caso não Homogéneo}

\begin{equation*}
    X' = AX + b(t) \Leftrightarrow
    \left[ \begin{matrix} x' \\ y' \end{matrix} \right] = 
    \left[ \begin{matrix} a & b \\ c & d \end{matrix} \right]
    \left[ \begin{matrix} x \\ y \end{matrix} \right] + 
    \left[ \begin{matrix} f(t) \\ g(t) \end{matrix} \right]
\end{equation*}

\begin{equation*}
    X(t) = e^{A(t - t_0)} \cdot X(t_0) + \int_{t_0}^{t} e^{A(t-s)} \cdot b(s) \, ds
\end{equation*}

\newpage

\subsection{Equações Diferenciais Lineares de Ordem Superior}

\begin{equation*}
    a_n y^{(n)} + a_{n-1} y^{(n-1)} + \ldots + a_0 y = f(t)
\end{equation*}

\subsubsection{Caso Homogéneo}

\begin{equation*}
    f(t) = 0 \ \ \ \ \Rightarrow \ \ \ \ y(t) = y_H
\end{equation*}

\begin{enumerate}
    \item Escrever na forma em que $a_n = 0$
    \item Escrever o \textbf{polinómio aniquilador}
        \begin{equation*}
            P(D)y = 0 \Leftrightarrow
            \left(D^n + a_{n-1} D^{n-1} + \ldots + a_0\right) y = 0 
        \end{equation*}
    \item Escrever o \textbf{polinómio característico}
        \begin{equation*}
            P(\lambda) = 0 \Leftrightarrow 
            P^n + a_{n-1} P^{n-1} + \ldots + a_0 = 0
        \end{equation*}
    \item Calcular as raízes do polinómio característico para encontrar as 
    \textbf{bases} das soluções
    \item Escrever a solução do problema como combinação linear das bases
\end{enumerate}

\subsubsection{Caso não Homogéneo}

\begin{equation*}
    f(t) \ne 0 \ \ \ \ \Rightarrow \ \ \ \ y(t) = y_H + y_P
\end{equation*}

\begin{enumerate}
    \item Calcular $y_H$ como se de um caso homogéneo se tratasse
    \item Determinar o \textbf{polinómio aniquilador} de $f(t)$
    \begin{equation*}
        P(D)f(t) = 0
    \end{equation*}
    \item Multiplicar ambos os lados da equação pelo polinómio, obtendo uma nova
    equação homogénea
    \item Calcular $y_P$ resolvendo a nova equação homogénea
    \item Juntar as duas soluções, verificando que não existe repetição de termos. \\
    Caso exista, multiplicar por $t,\,t^2,\ldots$
    \item Achar as constantes de $y_P$ pelo métodos dos coeficientes indeterminados
\end{enumerate}

TODO

\begin{center}
    \begin{tblr}[T]{colspec={ | Q[c,m] | Q[c,m] | Q[c,m] | Q[c,m] | }}
        \hline
        Raíz         & Multiplicidade & Base da Solução \\\hline
        $\lambda$    & $m$            &
        $\displaystyle \sum_{n=0}^{m-1} c\,t^n e^{\lambda t}$ \\\hline
        $a \pm b\,i$ & $m$            &
        $\displaystyle \sum_{n=0}^{m-1} c_1\,t^n e^{a t} \cos(b t) +
        c_2\,t^n e^{a t} \sin(b t)$ \\\hline
    \end{tblr}
\end{center}

\newpage

\subsection{Existência, Unicidade, Prolongamento de Soluções}

\subsubsection{Teorema de Picard}

\begin{equation*}
    \begin{cases}
        y' = f(x, y) \\
        y(x_o) = y_o
    \end{cases}
\end{equation*}

O Teorema de Picard garante que o problema tem \textbf{solução única} se 
$f(x, y)$ e $\displaystyle \frac{\partial f}{\partial y}$ são contínuas na 
vizinhança de $(x_o, y_o)$. \\

Caso alguma destas verificações falhe o teorema não é aplicável, sendo assim 
possível existirem várias soluções sem o contradizer.

\newpage

\section{Teoremas da Divergência e de Stokes}

\subsection{Superfícies em $\mathbb{R}^3$}

\subsubsection{Definição de Superfície}

TODO

\subsubsection{Reta Normal}

\begin{itemize}
    \item No caso de uma parametrização $g(u,v)$:
          \begin{equation*}
              \vec{N} = \frac{\partial g}{\partial u} \times \frac{\partial g}{\partial v}
              = \left\vert
              \begin{array}{c c c}
                  e_1                             & e_2 & e_3 \\[2pt]
                  \frac{\partial g_1}{\partial u} &
                  \frac{\partial g_2}{\partial u} &
                  \frac{\partial g_3}{\partial u}             \\[4pt]
                  \frac{\partial g_1}{\partial v} &
                  \frac{\partial g_2}{\partial v} &
                  \frac{\partial g_3}{\partial v}
              \end{array}
              \right\vert
          \end{equation*}
    \item No caso de um conjunto de nível $G(x, y, z) = 0$:
          \begin{equation*}
              \vec{N} = \vec{\nabla} G
          \end{equation*}
\end{itemize}

O \textbf{vetor normal unitário} é dado por:
\begin{equation*}
    \vec{n} = \frac{\vec{N}}{||\vec{N}||}
\end{equation*}

\subsubsection{Plano Tangente}

A equação de um plano tangente a uma superfície num ponto $P = (x_0, y_0, z_0)$ é dada por:
\begin{equation*}
    \vec{N} \cdot \left(x - x_0,\ y - y_0,\ z - z_0\right) = 0
\end{equation*}

\subsection{Integrais de Superfície}

\begin{equation*}
    \iint_S f(x, y, z) \ dS = \iint_D f(g(u, v))
    \left\lVert
    \frac{\partial g}{\partial u} \times
    \frac{\partial g}{\partial v}
    \right\rVert \, du\,dv
\end{equation*}

\begin{itemize}
    \item Área
          \begin{equation*}
              A = \iint_S \, dS
          \end{equation*}
    \item Massa
          \begin{equation*}
              M = \iint_S \sigma \, dS
          \end{equation*}
    \item Centro de Massa:
          \begin{equation*}
              \bar{x} = \frac{1}{M} \iint_S x(g(u, v)) \, \sigma \, dS
              \text{ \ \ \ \ \  (coordenada $x$)}
          \end{equation*}
\end{itemize}

\subsection{Operadores Diferenciais}

\subsubsection{Divergência}

\begin{align*}
    \text{div} F = \vec{\nabla} \cdot F & = \left(\frac{\partial}{\partial x}
    ,\ \frac{\partial}{\partial y}
    ,\ \frac{\partial}{\partial z}\right)
    \cdot (F_1,\ F_2,\ F_3)
    = \frac{\partial F_1}{\partial x}
    + \frac{\partial F_2}{\partial y}
    + \frac{\partial F_3}{\partial z}
\end{align*}

\subsubsection{Rotacional}

\begin{align*}
    \text{rot} F = \vec{\nabla} \times F & =
    \left\vert
    \begin{array}{c c c}
        e_1                         & e_2 & e_3 \\[2pt]
        \frac{\partial}{\partial x} &
        \frac{\partial}{\partial y} &
        \frac{\partial}{\partial z}             \\[2pt]
        F_1                         & F_2 & F_3
    \end{array}
    \right\vert
    = \left(\frac{\partial F_3}{\partial y} - \frac{\partial F_2}{\partial z}
    ,\ \frac{\partial F_1}{\partial z} - \frac{\partial F_3}{\partial x}
    ,\ \frac{\partial F_2}{\partial x} - \frac{\partial F_1}{\partial y}
    \right)
\end{align*}

\subsection{Fluxo de um Campo Vetorial}

\begin{equation*}
    \iint_S F \cdot \vec{n} \, dS =
    \iint_D F(g(u, v)) \cdot
    \left(
    \frac{\partial g}{\partial u} \times \frac{\partial g}{\partial v}
    \right) \, du\,dv
\end{equation*}

\begin{enumerate}
    \item Parametrizar a superfície
\end{enumerate}

\subsubsection{Teorema de Divergência}

Seja $S$ a \textbf{superfície fechada}, orientada \textbf{positivamente} (p/
fora), fronteira de uma região sólida $E$.

\begin{equation*}
    \iint_S \vec{F} \cdot \vec{n} \, dS =
    \iiint_E \text{div} \vec{F} \, dV
\end{equation*}

\subsection{Teorema de Stokes}

Seja $S$ uma \textbf{superfície orientada}, cuja fronteira é formada por uma
curva $C$ fechada e com orientação positiva.

\begin{equation*}
    \oint_C \vec{F} \ dg =
    \iint_S \text{rot} \vec{F} \ dS =
    \iint_D \text{rot} \vec{F} \cdot \vec{N} \, du\,dv
\end{equation*}

\subsubsection{Trabalho}
\begin{equation*}
    W = \oint_C F \, dg =
    \int_{a}^{b} F\left(g(t)\right) \cdot g'(t) \, dt
\end{equation*}

\subsubsection{Potencial Vetorial}

Diz-se que $\vec{G}$ é potencial vetorial de $\vec{F}$ se $\vec{F} =$ rot\,$\vec{G}$

\begin{enumerate}
    \item Verificar que div\,$\vec{F} = 0$
    \item Resolver a equação $\vec{F} =$ rot\,$\vec{G}$:
          \begin{align*}
              \left(\vec{F_1},\,\vec{F_2},\,\vec{F_3}\right) & = \text{rot}\,\vec{G}
              \\ \Leftrightarrow
              \left(\vec{F_1},\,\vec{F_2},\,\vec{F_3}\right) & =
              \left(\frac{\partial G_3}{\partial y} - \frac{\partial G_2}{\partial z}
              ,\,\frac{\partial G_1}{\partial z} - \frac{\partial G_3}{\partial x}
              ,\,\frac{\partial G_2}{\partial x} - \frac{\partial G_1}{\partial y}
              \right)
          \end{align*}
    \item Assumir que uma das componentes de $G$ é nulo, (ex. $G_2$):
          \begin{equation*}
              \left(\vec{F_1},\,\vec{F_2},\,\vec{F_3}\right) =
              \left(\frac{\partial G_3}{\partial y}
              ,\,\frac{\partial G_1}{\partial z} - \frac{\partial G_3}{\partial x}
              ,\, - \frac{\partial G_1}{\partial y}
              \right)
          \end{equation*}
    \item Integrar as outras componente em ordem à variável corresponde ao
          componente anulado (ex. $y$):
          \begin{equation*}
              \int \vec{F_1} \,dy = \int \frac{\partial G_3}{\partial y} \,dy
          \end{equation*}
          \begin{equation*}
              \int \vec{F_3} \,dy = \int -\frac{\partial G_1}{\partial y} \,dy
          \end{equation*}
\end{enumerate}

\newpage

\section{Transformadas de Laplace}

\begin{equation*}
    \mathcal{L}\left\{f(t)\right\} = F(s) = 
    \int_{0}^{+\infty} e^{-st}\cdot f(t) \, dt
\end{equation*}

\subsection{Tabela}

\begin{center}
    \begin{tblr}[T]{colspec={ Q[c,m] | Q[c,m] }}
        \hline
        $f(t) = \mathcal{L}^{-1}\left\{F(s)\right\}$ & 
        $F(s) = \mathcal{L}^{-1}\left\{f(t)\right\} \ , \ s > \alpha$ \\\hline
        $1$    & $m$            \\\hline
        $e^{at}$ & $m$            \\\hline
        $e^{at}$ & $m$            \\\hline
        $e^{at}$ & $m$            \\\hline
        $e^{at}$ & $m$            \\\hline
        $e^{at}$ & $m$            \\\hline
        
    \end{tblr}
\end{center}

\subsection*{Função Heaviside}

\begin{equation*}
    H_a(t) = H(t-a) = 
    \begin{cases}
        1, \ t > a \\
        0, \ t < a
    \end{cases}
\end{equation*}

\begin{equation*}
    1 - H_a(t) = 
    \begin{cases}
        0, \ t > a \\
        1, \ t < a
    \end{cases}
\end{equation*}

\begin{equation*}
    H_a(t) - H_b(t) = 
    \begin{cases}
        0, \ t < a \\
        1, \ a < t < b \\
        0, \ t > b
    \end{cases}
\end{equation*}

\newpage

\section{Séries de Fourier}

Sendo $f(x)$ uma função $f$ com extensão periódica $\bar{f}$ com período $2L$:

\begin{equation*}
    SFf(x) = \frac{a_0}{2} + \sum_{n=1}^{+\infty}
    \left(
    a_n \cdot \cos\left(\frac{n\pi x}{L}\right) +
    b_n \cdot \sin\left(\frac{n\pi x}{L}\right)
    \right)
\end{equation*}

\begin{equation*}
    a_n =
    \frac{1}{L} \int_{-L}^{L} f(x) \cdot \cos\left(\frac{n\pi x}{L}\right) \, dx
    \ \ \ \Rightarrow \ \ \
    a_0 =
    \frac{1}{L} \int_{-L}^{L} f(x) \, dx
\end{equation*}

\begin{equation*}
    b_n =
    \frac{1}{L} \int_{-L}^{L} f(x) \cdot \sin\left(\frac{n\pi x}{L}\right) \, dx
\end{equation*}

\begin{center}
    \begin{tblr}[T]{colspec={ | Q[c,m] | Q[c,m] | Q[c,m] | }}
        \hline
        $\cos(x) = - \cos(x)$ (par)    &
        $\cos(n\pi) = (-1)^n$          &
        $\displaystyle \int_a^b \cos(*x) = \left[\frac{\sin(*x)}{*}\right]_a^b$ \\\hline
        $\sin(x) = - \sin(-x)$ (ímpar) &
        $\sin(n\pi) = 0$               &
        $\displaystyle \int_a^b \sin(*x) = \left[-\frac{\cos(*x)}{*}\right]_a^b$
        \\\hline
    \end{tblr}
\end{center}

\subsection{Teorema da Convergência Pontual}

\begin{equation*}
    SFf(x) =
    \begin{cases}
        f(x)                     & \text{se $x$ é um ponto de continuidade de $f$}    \\
        \frac{f(x^+)+f(x^-)}{2}  & \text{se $x$ é um ponto de descontinuidade de $f$} \\
        \frac{f(-L^+)+f(L^-)}{2} & \text{se $x = -L$ ou $x = L$}                      \\
    \end{cases}
\end{equation*}

\subsection{Série de Senos}

Quando a função $f(x)$ é \textbf{ímpar}:

\begin{equation*}
    S_{\sin} f(x) =
    \sum_{n=1}^{+\infty}b_n \cdot \sin\left(\frac{n\pi x}{L}\right)
\end{equation*}

\begin{equation*}
    b_n =
    \frac{2}{L} \int_{0}^{L} f(x) \cdot \sin\left(\frac{n\pi x}{L}\right) \, dx
\end{equation*}

\subsection{Série de Cossenos}

Quando a função $f(x)$ é \textbf{par}:

\begin{equation*}
    S_{\cos} f(x) = \frac{a_0}{2} + \sum_{n=1}^{+\infty}
    a_n \cdot \cos\left(\frac{n\pi x}{L}\right)
\end{equation*}

\begin{equation*}
    a_n =
    \frac{2}{L} \int_{0}^{L} f(x) \cdot \cos\left(\frac{n\pi x}{L}\right) \, dx
    \ \ \ \Rightarrow \ \ \
    a_0 =
    \frac{2}{L} \int_{0}^{L} f(x) \, dx
\end{equation*}

\subsubsection*{Integral por partes}

\begin{equation*}
    \int_a^b uv' = [uv] _a^b - \int_a^b u'v
\end{equation*}

\newpage

\section{Equações Diferenciais Parciais}

\subsection{Método das Características (EDPs de 1ª Ordem)}

\begin{equation*}
    a(x,y)\frac{\partial u}{\partial x} + 
    b(x,y)\frac{\partial u}{\partial y} =
    c(x,y)
\end{equation*}

\subsubsection{Caso homogéneo}

\begin{equation*}
    c(x,y) = 0 \ \ \Rightarrow \ \ 
    a(x,y)\frac{\partial u}{\partial x} + 
    b(x,y)\frac{\partial u}{\partial y} = 0
\end{equation*}

\begin{enumerate}
    \item Resolver a EDO, isolando a constante $C$:
    \begin{equation*}
        \frac{dy}{dx} = \frac{b(x,y)}{a(x,y)} \ \ \Leftrightarrow \ \ 
        \ldots \ \ \Leftrightarrow \ \ C = g(x,y)
    \end{equation*}
    \item Escrever a solução:
    \begin{equation*}
        u(x,y) = f(C)
    \end{equation*}
\end{enumerate}

\subsubsection{Caso não homogéneo}

\begin{enumerate}
    \item Determinar as curvas características:
    \begin{enumerate}
        \item resolver a seguinte EDO, isolando a constante $C$:
        \begin{equation*}
            \frac{dy}{dx} = \frac{b(x,y)}{a(x,y)} \ \ \Leftrightarrow \ \ 
            \ldots \ \ \Leftrightarrow \ \ C = f(x,y)
        \end{equation*}
    \end{enumerate}
    \item Fazer uma mudança de varáveis $(x,y) \rightarrow (s,r)$
    \begin{enumerate}
        \item Escolher $s = C = f(x,y)$
        \item Escolher $r$ tal que o jacobiano da transformação seja \textbf{não nulo}:
        \begin{equation*}
            J(s, r) =
            \left\lvert
            \begin{array}{l l}
                \frac{\partial s}{\partial x} &
                \frac{\partial s}{\partial y} \vspace{2pt} \\
                \frac{\partial r}{\partial x} &
                \frac{\partial r}{\partial y}
            \end{array}
            \right\rvert \ne 0
        \end{equation*}
        \item Utilizar a regra da cadeia para simplificar 
        $\frac{\partial u}{\partial x}$ e $\frac{\partial u}{\partial y}$:
        \begin{itemize}
            \item $\displaystyle \frac{\partial u}{\partial x} = 
            \frac{\partial u}{\partial s} \frac{\partial s}{\partial x} + 
            \frac{\partial u}{\partial r} \frac{\partial r}{\partial x}$
            \item $\displaystyle \frac{\partial u}{\partial y} = 
            \frac{\partial u}{\partial s} \frac{\partial s}{\partial y} + 
            \frac{\partial u}{\partial r} \frac{\partial r}{\partial y}$
        \end{itemize}
    \end{enumerate}
    \item Substituir na equação original e simplificar, \textbf{deve obter-se uma EDO}:
        \begin{equation*}
            a(s,r)\left(\frac{\partial u}{\partial s} 
            \frac{\partial s}{\partial x} + 
            \frac{\partial u}{\partial r} \frac{\partial r}{\partial x}\right) + 
            b(s,r)\left(\frac{\partial u}{\partial s} 
            \frac{\partial s}{\partial y} + 
            \frac{\partial u}{\partial r} \frac{\partial r}{\partial y}\right) =
            c(s,r)
        \end{equation*}
    \item Resolver a EDO obtida, obtendo-se uma solução na forma $u(s, r)$
    \item Inverter a mudança de variáveis:
    \begin{equation*}
        u(s, r) \rightarrow u(x, y)
    \end{equation*}
\end{enumerate}

\newpage

\subsection{Método de Separação de Variáveis (EDPs de 2ª Ordem)}

\begin{equation*}
    u(x, t) = X(x) \cdot T(t)
\end{equation*}
\begin{equation*}
    \text{Condição de fronteira (CF):\ \ } u(0, t) = u(\pi, t) = 0 \text{\ \ ($t$ variável)}
\end{equation*}
\begin{equation*}
    \text{Condição inicial (CI):\ \ } u(x, 0) = f(x) \text{\ \ ($t$ fixo $= 0$)}
\end{equation*}

\begin{enumerate}
    \item Substituir $u(x, t)$ na equação diferencial por $X(x)T(t)$
    \item Separar as variáveis dos dois lados da igualdade 
          (deixar $X$ \textbf{mais simples}):
          \begin{equation*}
              \frac{T'}{T} = \frac{X''}{X}
          \end{equation*}
    \item Igualar os dois lados da equação a lambda ($\lambda$):
          \begin{equation*}
              \frac{T'}{T} = \frac{X''}{X} = \lambda
          \end{equation*}
    \item Analisar as condições de fronteira, sabendo que $T(t) \neq 0$:
          \begin{equation*}
              u(t, 0) = u(t, \pi) = 0 \ \Rightarrow\ X(0) = X(\pi) = 0
          \end{equation*}
    \item Construir os dois problemas (EDOs)a resolver:
          \begin{equation*}
              \text{P1:\ \ \ }
              \begin{cases}
                  X'' - \lambda X = 0 \\
                  X(0) = X(\pi) = 0
              \end{cases}
              \text{se $x \in\ ]0, \pi[$}
          \end{equation*}
          \begin{equation*}
              \text{P2:\ \ \ } \frac{T'}{T} = \lambda
          \end{equation*}
    \item Resolver P1 testando as várias possibilidades para $\lambda$:
          \begin{itemize}
              \item $\lambda = 0$
              \item $\lambda > 0$
              \item $\lambda < 0$
          \end{itemize}
    \item Resolver P2 para os valores de $\lambda$ obtidos anteriormente
    \item Combinar os resultados obtidos, obtendo a solução do PVF:
          \begin{equation*}
              u_k(t, x) = T_k(t) X_k(t) \ \Rightarrow \
              u(t,x) = \sum_{k=1}^{\infty} c_k T_k X_k
          \end{equation*}
    \item Calcular as constantes $c_k$ utilizando a condição inicial:
          \begin{equation*}
              u(0,x) = \sum_{k=1}^{\infty} c_k X_k = \text{CI}
          \end{equation*}
    \item Substituir os valores obtidos em $u(t,x)$
\end{enumerate}

\newpage

\section{Extras}

\subsection{Mudança de Variáveis de Integração}

\subsubsection{Coordenadas Polares}

\begin{equation*}
    \iint f(r\cos\theta, r\sin\theta) \cdot r\,dr\,d\theta
\end{equation*}

\subsubsection{Coordenadas Cilíndricas}

\begin{equation*}
    \iiint f(r\cos\theta, r\sin\theta, z) \cdot r\,dz\,dr\,d\theta
\end{equation*}

\subsubsection{Coordenadas Esféricas}

\begin{equation*}
    \iiint f(r\sin\varphi\cos\theta, r\sin\varphi\sin\theta, r\cos\varphi) \cdot r^2\sin\varphi\,dr\,d\varphi\,d\theta
\end{equation*}

\end{document}