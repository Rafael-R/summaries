\documentclass[11pt, a4paper]{article}
\usepackage[margin=2cm]{geometry}
\usepackage{hyperref}
\usepackage{enumitem}
\usepackage{amsmath,amsfonts,amssymb, amstext}
\usepackage{array}
\usepackage{cancel}

\newcolumntype{C}{>{$}c<{$}}

\hypersetup{
    colorlinks,
    citecolor=black,
    filecolor=black,
    linkcolor=black,
    urlcolor=black
}

\setlength{\parindent}{0pt}

\begin{document}

\begin{titlepage}
    \begin{center}
        \vspace*{5cm}

        \textbf{\LARGE Cálculo Diferencial e Integral III}
        \vspace{1cm}

        \Large Resumo
        \vspace{2cm}

        \textbf{Rafael Rodrigues}
        \vfill
        LEIC \\
        Instituto Superior Técnico \\
        2023/2024
    \end{center}
\end{titlepage}

\tableofcontents

\newpage


\section{Equações Diferenciais Ordinárias}

\subsection{Equações Escalares de 1ª Ordem}

\subsubsection{Equações Lineares [1]}

\begin{equation*}
    \frac{dy}{dt} + a(t)y = b(t)
\end{equation*}

\begin{enumerate}
    \item Verificar que está escrita na forma linear
    \item Determinar o fator integrante: $\displaystyle \mu(t) = \exp\left(\int a(t) \, dt \right)$
    \item Determinar $y$:
          $\displaystyle y(t) = \frac{1}{\mu(t)}\left[\int \mu(t)b(t) \, dt + C\right]$
\end{enumerate}

\subsubsection{Equações Separáveis [2]}

\begin{equation*}
    f(y)\frac{dy}{dt} = g(t) \Leftrightarrow
    f(y) \, dy = g(t) \, dt
\end{equation*}

\begin{enumerate}
    \item Verificar que está escrita na forma separável
    \item Integrar ambos os membros da igualdade:
          $\displaystyle \int f(y) \, dy = \int g(t) \, dt + C$
\end{enumerate}

\subsubsection{Equações Exactas [2]}

\begin{equation*}
    M(t,y) + N(t, y)\frac{dy}{dt} = 0 \Leftrightarrow
    M(t,y) \, dt + N(t, y) \, dy = 0
\end{equation*}

\begin{enumerate}
    \item Verificar se a equação é exata:
          $\displaystyle \frac{\partial M}{\partial y} =
              \frac{\partial N}{\partial t} \Rightarrow
              \exists\ \Phi: \mathbb{R}^2 \rightarrow \mathbb{R}$ tal que
          $\nabla \Phi(t, y) = (M, N)$
    \item Encontrar um potencial para $\Phi(t, y)$:
          \begin{enumerate}
              \item TODO
          \end{enumerate}
\end{enumerate}

\subsubsection{Equações Redutíveis a Exactas [2]}

TODO

\subsection{Equações Diferenciais Ordinárias de Ordem $n$}

\subsubsection{Equações Lineares de Ordem $n$ — Caso Homogéneo [3]}

TODO

\begin{enumerate}
    \item Colocar na
\end{enumerate}

\subsection{Equações Vetoriais de 1ª Ordem}

\subsubsection{Equações Vetoriais Lineares — Caso Homogéneo [3]}

TODO

\subsubsection{Equações Vetoriais Lineares — Caso não Homogéneo [4]}

TODO

Exercício 1 e 2 da Aula Prática 5

\subsubsection{Exponencial de uma Matriz [4]}

TODO

Exercício 1 da Aula Prática 4

Verificação:
\begin{itemize}
    \item $\displaystyle e^{At}|_{t=0} = I$
    \item $\displaystyle \frac{d}{dt}e^{At}|_{t=0} = A $
\end{itemize}

\subsection{Equações Lineares de Ordem $n$ — Caso não Homogéneo [4]}

TODO

Exercícios 1 e 2 da Aula Prática 5

\subsection{Existência, Unicidade, Prolongamento de Soluções [5]}

TODO

Exercícios 3, 4 e 5 da Aula Prática 5 \\
Exercícios 1 e 2 da Aula Prática 6 (Ficha 5b)

\newpage


\section{Teoremas da Divergência e de Stokes}

\subsection{Superfícies em $\mathbb{R}^3$ [6]}

TODO

Exercício 3 da Aula Prática 6 (Ficha 5b)

\subsubsection{Definição de Superfície}

\subsubsection{Reta Normal}

\begin{itemize}
    \item No caso de uma parametrização $g(u,v)$:
          \begin{equation*}
              \vec{N} = \frac{\partial g}{\partial u} \times \frac{\partial g}{\partial v}
              = \left\vert
              \begin{array}{c c c}
                  e_1                             & e_2 & e_3 \\[2pt]
                  \frac{\partial g_1}{\partial u} &
                  \frac{\partial g_2}{\partial u} &
                  \frac{\partial g_3}{\partial u}             \\[4pt]
                  \frac{\partial g_1}{\partial v} &
                  \frac{\partial g_2}{\partial v} &
                  \frac{\partial g_3}{\partial v}
              \end{array}
              \right\vert
          \end{equation*}
    \item No caso de um conjunto de nível $G(x, y, z) = 0$:
          \begin{equation*}
              \vec{N} = \vec{\nabla} G
          \end{equation*}
\end{itemize}

O \textbf{vetor normal unitário} é dado por:
\begin{equation*}
    \vec{n} = \frac{\vec{N}}{||\vec{N}||}
\end{equation*}

\subsubsection{Plano Tangente}

A equação de um plano tangente a uma superfície num ponto $P = (x_0, y_0, z_0)$ é dada por:
\begin{equation*}
    \vec{N} \cdot \left(x - x_0,\ y - y_0,\ z - z_0\right) = 0
\end{equation*}

\subsection{Integrais de Superfície [7]}

TODO

\subsection{Operadores Diferenciais [7]}

\subsubsection{Divergência}

\begin{align*}
    \text{div} F = \vec{\nabla} \cdot F & = \left(\frac{\partial}{\partial x}
    ,\ \frac{\partial}{\partial y}
    ,\ \frac{\partial}{\partial z}\right)
    \cdot (F_1,\ F_2,\ F_3)
    = \frac{\partial F_1}{\partial x}
    + \frac{\partial F_2}{\partial y}
    + \frac{\partial F_3}{\partial z}
\end{align*}

\subsection{Rotacional}

\begin{align*}
    \text{rot} F = \vec{\nabla} \times F & =
    \left\vert
    \begin{array}{c c c}
        e_1                         & e_2 & e_3 \\[2pt]
        \frac{\partial}{\partial x} &
        \frac{\partial}{\partial y} &
        \frac{\partial}{\partial z}             \\[2pt]
        F_1                         & F_2 & F_3
    \end{array}
    \right\vert
    = \left(\frac{\partial F_3}{\partial y} - \frac{\partial F_2}{\partial z}
    ,\ \frac{\partial F_1}{\partial z} - \frac{\partial F_3}{\partial x}
    ,\ \frac{\partial F_2}{\partial x} - \frac{\partial F_1}{\partial y}
    \right)
\end{align*}

\subsection{Fluxo de um Campo Vetorial}

TODO
\begin{equation*}
    \iint_{S} F \cdot \vec{n} \ dS =
    \iint_{T} F(g(u, v)) \cdot
    \left(
    \frac{\partial g}{\partial u} \times \frac{\partial g}{\partial v}
    \right) \ dudv
\end{equation*}

\subsubsection{Teorema de Divergência [8]}

TODO
\begin{equation*}
    \iint_{S} F \cdot \vec{n} \ dS =
    \iiint_{V} \text{div} F \ dV
\end{equation*}

\subsection{Teorema de Stokes [9]}

TODO

\subsubsection{Potencial Vetorial}

TODO

\newpage


\section{Introdução às Equações Diferenciais Parciais}

\subsection{Método de Separação de Variáveis [10]}

TODO

\subsection{Séries de Fourier [10]}

TODO

\subsection{[11]}

\end{document}