\documentclass[11pt, a4paper]{article}
\usepackage[margin=2cm]{geometry}
\usepackage{hyperref}
\usepackage{enumitem}
\usepackage{amsmath,amsfonts,amssymb, amstext}
\usepackage{array}
\usepackage{cancel}

\newcolumntype{C}{>{$}c<{$}}

\hypersetup{
    colorlinks,
    citecolor=black,
    filecolor=black,
    linkcolor=black,
    urlcolor=black
}

\setlength{\parindent}{0pt}

\begin{document}

\begin{titlepage}
    \begin{center}
        \vspace*{5cm}

        \textbf{\LARGE Cálculo Diferencial e Integral II}
        \vspace{1cm}

        \Large Resumo
        \vspace{2cm}

        \textbf{Rafael Rodrigues}
        \vfill
        LEIC \\
        Instituto Superior Técnico \\
        2023/2024
    \end{center}
\end{titlepage}

\tableofcontents

\setcounter{section}{1}

\newpage
\section{Limites e Continuidade}

TODO

\section{Diferenciabilidade}

Derivada de uma função $f$ segundo um vetor $v$ no ponto $a$:

\begin{equation*}
    \frac{\partial f}{\partial v}(a) = D_vf(a) =
    \lim_{t \to 0} \frac{f(a + tv) - f(a)}{t}
\end{equation*}

Uma função $f$ diz-se \textbf{diferenciável} num ponto $a$ se:

\begin{equation*}
    \lim_{x \to a} \frac{f(x) - f(a) - Df(a) \cdot (x - a)}{||x - a||} = 0
\end{equation*}

Se $f$ é diferenciável em $a$ então:

\begin{equation*}
    D_vf(a) = Df(a) \cdot v
\end{equation*}

\subsection{Matriz Jacobiana}

Para uma função $f: \mathbb{R}^n \to \mathbb{R}^m$:

\begin{equation*}
    D_f(x_1, \ldots, x_n) =
    \left[
        \begin{array}{c c c}
            \frac{\partial f_1}{\partial x_1} & \cdots &
            \frac{\partial f_1}{\partial x_n} \vspace{2pt}                   \\
            \vdots                            & \ddots & \vdots \vspace{2pt} \\
            \frac{\partial f_m}{\partial x_1} & \cdots &
            \frac{\partial f_m}{\partial x_n}
        \end{array}
        \right]
\end{equation*}

\subsection{Gradiente}

\begin{equation*}
    \nabla f(x, y) = \left(\frac{\partial f}{\partial x}, \frac{\partial f}{\partial y}\right)
\end{equation*}

\subsection{Classe $C^1$}

Uma função $f: \mathbb{R}^n \to \mathbb{R}^m$ diz-se de classe $C^1$ se as
suas derivadas parciais são contínuas.

Se uma função $f$ é de classe $C^1$ em $a$ então $f$ é \textbf{diferenciável} em
$a$.

\section{Derivada da Função Composta}

\subsection{Cálculo da derivada}

\begin{equation*}
    D(g \circ f)(a) = Dg\left(f(a)\right) \cdot Df(a)
\end{equation*}

\subsection{Regra da Cadeia}

TODO

\section{Derivadas de Ordem Superior e Extremos}

\subsection{Pontos de estacionaridade (ou críticos)}

\begin{equation*}
    \nabla f(x, y) = (0, 0) \Leftrightarrow
    \begin{cases}
        \frac{\partial f}{\partial x} = 0 \vspace{2pt} \\
        \frac{\partial f}{\partial y} = 0
    \end{cases}
\end{equation*}

\subsection{Matriz Hessiana}

\begin{equation*}
    H_f(x, y) =
    \left[
    \begin{array}{l l}
        \frac{\partial^2 f}{\partial x^2}          &
        \frac{\partial^2 f}{\partial y \partial x} \vspace{2pt} \\
        \frac{\partial^2 f}{\partial x \partial y} &
        \frac{\partial^2 f}{\partial y^2}
    \end{array}
    \right]
\end{equation*}

Classificação dos pontos críticos:
\begin{itemize}
    \item det $H_f(x,y) < 0 \ \rightarrow$ ponto em sela (indefinida)
    \item det $H_f(x,y) > 0 \ \rightarrow$ extremo
          \begin{itemize}
              \item tr $H_f(x,y) > 0 \ \rightarrow (x,y)$ é um ponto mínimo (definida positiva)
              \item tr $H_f(x,y) < 0 \ \rightarrow (x,y)$ é um ponto máximo (definida negativa)
          \end{itemize}
    \item det $H_f(x,y) > 0 \ \rightarrow$ inconclusivo
          \begin{itemize}
              \item tr $H_f(x,y) \geq 0 \ \rightarrow (x,y)$ é um ponto mínimo ou ponto de sela (semi-definida positiva)
              \item tr $H_f(x,y) \leq 0 \ \rightarrow (x,y)$ é um ponto máximo ou ponto de sela (semi-definida negativa)
          \end{itemize}
\end{itemize}

\subsection{Teorema de Weierstrass}

TODO Semana 6

Uma função $f$ num conjunto compacto $S$ tem máximo e mínimo.

\section{Função Inversa e Função Implícita}

\subsection*{Jacobiano}

\begin{equation*}
    J_f(x, y) =
    \left\lvert
    \begin{array}{l l}
        a = \frac{\partial f_1}{\partial x} &
        b = \frac{\partial f_1}{\partial y} \vspace{2pt} \\
        c = \frac{\partial f_2}{\partial x} &
        d = \frac{\partial f_2}{\partial y}
    \end{array}
    \right\rvert =
    ad - cb
\end{equation*}

\subsection{Teorema da Função Inversa}

Mostrar que $f$ é localmente invertível em torno dum ponto $(a, b)$:

\begin{enumerate}
    \item Calcular a matriz jacobiana $Df$
    \item Calcular o determinante de $Df(a, b)$
    \item Se det $Df(a, b) \ne 0$ então $f$ tem inversa local $C^1$ em torno desse ponto
\end{enumerate}

Calcular a derivada da função inversa num ponto $(c, d)$:

\begin{enumerate}
    \item Verificar que $(c, d) = F(a, b)$
    \item Inverter a matriz jacobiana $Df(a, b)$
\end{enumerate}

\begin{equation*}
    Df^{-1}\left(f(a, b)\right) = Df^{-1}(c, d) = \left[Df(a, b)\right]^{-1}
\end{equation*}

\subsection{Teorema da Função Implícita}

Mostrar que uma função $F(x_i, \ldots, x_j) = 0$ define $x_i$ como função de $x_j$ $\left[x_i = f(x_j)\right]$ num ponto $(a, b)$

\begin{enumerate}
    \item Verificar que $F(a, b) = 0$
    \item Calcular $DF$
    \item Se $D_{x_i}F(a, b) \ne 0$ então $F$ determina
          $x_i = f(x_j)$ num ponto $(a, b)$
\end{enumerate}

Calcular a derivada da função implícita num ponto $b$:

\begin{equation*}
    D_{x_i}f (b) = - \left[D_{x_i}F(a, b)\right]^{-1} \cdot D_{x_j}F(a, b)
\end{equation*}

\section{Extremos Condicionados}

\subsection{Multiplicadores de Lagrange}

TODO

\begin{equation*}
    \begin{cases}
        \nabla f = \sum_{i}^{n} \lambda_i \nabla F_i \\
        F_1 = 0                                      \\
        \ \ \ \ \ \vdots                             \\
        F_n = 0
    \end{cases}
\end{equation*}

\section{Teorema de Fubini}

\subsection{Aplicações do Integral}

\subsubsection{Volume e Centróide}

O volume de um sólido é dado por:

\begin{equation*}
    V_S = \int_{S} 1
\end{equation*}

O seu \textbf{centróide} na coordenada $x_i$ é dado por:

\begin{equation*}
    \overline{x_i} = \frac{1}{V_S} \cdot \int_{S} x_i
\end{equation*}

\subsubsection{Massa de Sólido e Centro de massa}

Para um sólido $S$ e uma \textbf{função de densidade de massa} $f$:

\begin{equation*}
    M_S = \int_{S} f
\end{equation*}

O \textbf{centro de massa} na coordenada $x_i$ é dado por:

\begin{equation*}
    \overline{x_i} = \frac{1}{M_S} \cdot \int_{S} x_i \, f
\end{equation*}

\subsubsection{Momento de Inércia}

O momento de inércia em relação a um eixo $L$ é dado por:

\begin{equation*}
    I_L = \int_{S} f \cdot \left(\text{distância à lateral}\right)^2
\end{equation*}

Por exemplo para o eixo $xx$ teríamos:

\begin{equation*}
    I_x = \int_{S} f \cdot \left(y^2 + z^2\right)
\end{equation*}


\section{Mudança de Variáveis de Integração}

\subsection{Coordenadas Polares}

\begin{equation*}
    \iint f\left(r\cos\theta, r\sin\theta\right) \cdot r\,dr\,d\theta
\end{equation*}

\subsection{Coordenadas Cilíndricas}

\begin{equation*}
    \iiint f\left(r\cos\theta, r\sin\theta, z\right) \cdot r\,dz\,dr\,d\theta
\end{equation*}

\subsection{Coordenadas Esféricas}

\begin{equation*}
    \iiint f\left(r\sin\varphi\cos\theta, r\sin\varphi\sin\theta, r\cos\varphi\right) \cdot r^2\sin\varphi\,dr\,d\varphi\,d\theta
\end{equation*}

\section{Teorema Fundamental do Cálculo e Regra de Leibniz}

\subsection{Teorema Fundamental do Cálculo}

\begin{equation*}
    F(x) = \int_{a(x)}^{b(x)} f(t) \, dt
\end{equation*}

\begin{equation*}
    F'(x) = f\left(b(x)\right) \cdot b'(x) - f\left(a(x)\right) \cdot a'(x)
\end{equation*}

\subsection{Regra de Leibniz}

\begin{equation*}
    F(x) = \int_{a}^{b} f(x, t) \, dt
\end{equation*}

\begin{equation*}
    F'(x) = \int_{a}^{b} \frac{\partial f}{\partial x}(x, t) \, dt
\end{equation*}

\section{Integrais de Linha}

\begin{enumerate}
    \item Parametrizar em função de $t$, ou seja, obter $g(t)$
    \item Obter a derivada de $g$, ou seja, $g'(t) = \left\langle x'(t), y'(t)\right\rangle$
    \item No caso do \textbf{campo escalar}, calcular $\left\lVert g'(t) \right\rVert = \sqrt{\left(x'(t)\right)^2 + \left(y'(t)\right)^2}$
\end{enumerate}

\subsection{Campos Escalares}

\begin{equation*}
    \int_{C} f \, ds =
    \int_{a}^{b} f\left(g(t)\right) \cdot \left\lVert g'(t) \right\rVert  \, dt
\end{equation*}

\subsection{Campos Vetoriais}

\begin{equation*}
    W = \int_{C} F \, dg =
    \int_{a}^{b} F\left(g(t)\right) \cdot g'(t) \, dt
\end{equation*}

Trocar a orientação da curva troca o sinal do integral.

\section{Campos Fechados. Campos Gradientes. Teorema Fundamental do Cálculo}

\subsection{Teorema Fundamental do Cálculo para Integrais de Linha}

\begin{equation*}
    \int_{C} F \, dg =
    \int_{C} \nabla\varphi \, dg =
    \varphi\left(g(b)\right) - \varphi\left(g(a)\right)
\end{equation*}

\subsection{Campos Gradientes e Potenciais Escalares}

Um campo vetorial diz-se gradiente se $F = \nabla\varphi$, neste caso a $\varphi$
chama-se o \textbf{potencial escalar} de $F$.

\begin{equation*}
    \int_{C} \nabla\varphi \, dr = 0 \text{ , se a curva é fechada}
\end{equation*}

Para encontrar um potencial escalar:

\begin{equation*}
    \nabla\varphi = F \Leftrightarrow
    \begin{cases}
        \frac{\partial \varphi}{\partial x_1} = F_1 \\
        \ \ \ \ \ \ \vdots                          \\
        \frac{\partial \varphi}{\partial x_n} = F_n
    \end{cases} \Leftrightarrow
    \begin{cases}
        \varphi = \int F_1 \, dx_1 \\
        \ \ \ \, \vdots            \\
        \varphi = \int F_n \, dx_n
    \end{cases}
\end{equation*}

\subsection{Campos Fechados e Campos Gradientes}

Um campo vetorial diz-se fechado se $DF$ é simétrica:
\begin{equation*}
    \frac{\partial F_i}{\partial x_j} = \frac{\partial F_j}{\partial x_i}
    \ , \forall \  i \neq j
\end{equation*}

Se um campo vetorial \textbf{é gradiente}, então \textbf{também é fechado}.

\subsection{Condição Necessária e Suficiente para ser Gradiente}

\section{Homotopia e Teorema de Green}

TODO Semana 13

\begin{equation*}
    \int_{C} P \, dx + Q \, dy =
    \iint_{D} \frac{\partial Q}{\partial x} - \frac{\partial P}{\partial y} \, dA
\end{equation*}

\end{document}