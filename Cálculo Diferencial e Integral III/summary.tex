\documentclass[11pt, a4paper]{article}
\usepackage[margin=2cm]{geometry}
\usepackage{hyperref}
\usepackage{enumitem}
\usepackage{amsmath,amsfonts,amssymb, amstext}
\usepackage{tabularray}
\usepackage{array}
\usepackage{cancel}

\newcolumntype{C}{>{$}c<{$}}

\hypersetup{
    colorlinks,
    citecolor=black,
    filecolor=black,
    linkcolor=black,
    urlcolor=black
}

\setlength{\parindent}{0pt}

\begin{document}

\begin{titlepage}
    \begin{center}
        \vspace*{5cm}

        \textbf{\LARGE Cálculo Diferencial e Integral III}
        \vspace{1cm}

        \Large Resumo
        \vspace{2cm}

        \textbf{Rafael Rodrigues}
        \vfill
        LEIC \\
        Instituto Superior Técnico \\
        2023/2024
    \end{center}
\end{titlepage}

\tableofcontents

\newpage


\section{Equações Diferenciais Ordinárias}

\subsection{Equações Escalares de 1ª Ordem}

\subsubsection{Equações Lineares}

\begin{equation*}
    \frac{dy}{dt} + a(t)y = b(t)
\end{equation*}

\begin{enumerate}
    \item Verificar que está escrita na forma linear
    \item Determinar o fator integrante: $\displaystyle \mu(t) = \exp\left(\int a(t) \, dt \right)$
    \item Determinar $y$:
          $\displaystyle y(t) = \frac{1}{\mu(t)}\int \mu(t)b(t) \, dt + C$
\end{enumerate}

\subsubsection{Equações Separáveis}

\begin{equation*}
    f(y)\frac{dy}{dt} = g(t) \Leftrightarrow
    f(y) \, dy = g(t) \, dt
\end{equation*}

\begin{enumerate}
    \item Verificar que está escrita na forma separável
    \item Integrar ambos os membros da igualdade:
          $\displaystyle \int f(y) \, dy = \int g(t) \, dt + C$
\end{enumerate}

\subsubsection{Equações Exatas}

\begin{equation*}
    M(t,y) + N(t, y)\frac{dy}{dt} = 0 \Leftrightarrow
    M(t,y) \, dt + N(t, y) \, dy = 0
\end{equation*}

\begin{enumerate}
    \item Verificar se a equação é exata:
          $\displaystyle \frac{\partial M}{\partial y} =
              \frac{\partial N}{\partial t} \Rightarrow
              \exists\ \Phi: \mathbb{R}^2 \rightarrow \mathbb{R}$ tal que
          $\nabla \Phi(t, y) = (M, N)$
    \item Encontrar um potencial para $\Phi(t, y)$:
          \begin{equation*}
              \begin{cases}
                  \Phi = \int M \,dx \\
                  \Phi = \int N \,dy
              \end{cases} \Rightarrow
              \Phi = C
          \end{equation*}
    \item Isolar o $y$
\end{enumerate}

\subsubsection{Equações Redutíveis a Exatas}

\begin{equation*}
    M(t,y) + N(t, y)\frac{dy}{dt} = 0
\end{equation*}

\begin{enumerate}
    \item Quando a equação não é exata:
          $\displaystyle \frac{\partial M}{\partial y} \neq
              \frac{\partial N}{\partial t}$
    \item Calcular as razões:
          \begin{equation*}
              \frac{M_y - N_t}{N} \text{\ \ \ \ e \ \ \ } \frac{N_t - M_y}{M}
          \end{equation*}
    \item Escolher a razão que depender apenas de uma variável e calcular o
          \textbf{fator integrante}:
          \begin{equation*}
              \mu = \exp\left(\int \text{razão escolhida}\right)
          \end{equation*}
    \item Resolver a \textbf{equação exata}:
          $\displaystyle \mu M(t,y) + \mu N(t, y)\frac{dy}{dt} = 0$

\end{enumerate}

\subsection{Equações Diferenciais Ordinárias de Ordem $n$}

\subsubsection{Equações Lineares de Ordem $n$ — Caso Homogéneo}

TODO

\begin{enumerate}
    \item Colocar na
\end{enumerate}

\subsection{Equações Vetoriais de 1ª Ordem}

\subsubsection{Equações Vetoriais Lineares — Caso Homogéneo}

TODO

\subsubsection{Equações Vetoriais Lineares — Caso não Homogéneo}

TODO

Exercício 1 e 2 da Aula Prática 5

\subsubsection{Exponencial de uma Matriz}

TODO

Exercício 1 da Aula Prática 4

Verificação:
\begin{itemize}
    \item $\displaystyle e^{At}|_{t=0} = I$
    \item $\displaystyle \frac{d}{dt}e^{At}|_{t=0} = A $
\end{itemize}

\subsection{Equações Lineares de Ordem $n$ — Caso não Homogéneo}

TODO

Exercícios 1 e 2 da Aula Prática 5

\subsection{Existência, Unicidade, Prolongamento de Soluções}

TODO

Exercícios 3, 4 e 5 da Aula Prática 5 \\
Exercícios 1 e 2 da Aula Prática 6 (Ficha 5b)

\newpage

\section{Teoremas da Divergência e de Stokes}

\subsection{Superfícies em $\mathbb{R}^3$}

\subsubsection{Definição de Superfície}

TODO

\subsubsection{Reta Normal}

\begin{itemize}
    \item No caso de uma parametrização $g(u,v)$:
          \begin{equation*}
              \vec{N} = \frac{\partial g}{\partial u} \times \frac{\partial g}{\partial v}
              = \left\vert
              \begin{array}{c c c}
                  e_1                             & e_2 & e_3 \\[2pt]
                  \frac{\partial g_1}{\partial u} &
                  \frac{\partial g_2}{\partial u} &
                  \frac{\partial g_3}{\partial u}             \\[4pt]
                  \frac{\partial g_1}{\partial v} &
                  \frac{\partial g_2}{\partial v} &
                  \frac{\partial g_3}{\partial v}
              \end{array}
              \right\vert
          \end{equation*}
    \item No caso de um conjunto de nível $G(x, y, z) = 0$:
          \begin{equation*}
              \vec{N} = \vec{\nabla} G
          \end{equation*}
\end{itemize}

O \textbf{vetor normal unitário} é dado por:
\begin{equation*}
    \vec{n} = \frac{\vec{N}}{||\vec{N}||}
\end{equation*}

\subsubsection{Plano Tangente}

A equação de um plano tangente a uma superfície num ponto $P = (x_0, y_0, z_0)$ é dada por:
\begin{equation*}
    \vec{N} \cdot \left(x - x_0,\ y - y_0,\ z - z_0\right) = 0
\end{equation*}

\subsection{Integrais de Superfície}

\begin{equation*}
    \iint_S f(x, y, z) \ dS = \iint_D f(g(u, v))
    \left\lVert
    \frac{\partial g}{\partial u} \times
    \frac{\partial g}{\partial v}
    \right\rVert \, du\,dv
\end{equation*}

\begin{itemize}
    \item Área
          \begin{equation*}
              A = \iint_S \, dS
          \end{equation*}
    \item Massa
          \begin{equation*}
              M = \iint_S \sigma \, dS
          \end{equation*}
    \item Centro de Massa:
          \begin{equation*}
              \bar{x} = \frac{1}{M} \iint_S x(g(u, v)) \, \sigma \, dS
              \text{ \ \ \ \ \  (coordenada $x$)}
          \end{equation*}
\end{itemize}

\subsection{Operadores Diferenciais}

\subsubsection{Divergência}

\begin{align*}
    \text{div} F = \vec{\nabla} \cdot F & = \left(\frac{\partial}{\partial x}
    ,\ \frac{\partial}{\partial y}
    ,\ \frac{\partial}{\partial z}\right)
    \cdot (F_1,\ F_2,\ F_3)
    = \frac{\partial F_1}{\partial x}
    + \frac{\partial F_2}{\partial y}
    + \frac{\partial F_3}{\partial z}
\end{align*}

\subsubsection{Rotacional}

\begin{align*}
    \text{rot} F = \vec{\nabla} \times F & =
    \left\vert
    \begin{array}{c c c}
        e_1                         & e_2 & e_3 \\[2pt]
        \frac{\partial}{\partial x} &
        \frac{\partial}{\partial y} &
        \frac{\partial}{\partial z}             \\[2pt]
        F_1                         & F_2 & F_3
    \end{array}
    \right\vert
    = \left(\frac{\partial F_3}{\partial y} - \frac{\partial F_2}{\partial z}
    ,\ \frac{\partial F_1}{\partial z} - \frac{\partial F_3}{\partial x}
    ,\ \frac{\partial F_2}{\partial x} - \frac{\partial F_1}{\partial y}
    \right)
\end{align*}

\subsection{Fluxo de um Campo Vetorial}

\begin{equation*}
    \iint_S F \cdot \vec{n} \, dS =
    \iint_D F(g(u, v)) \cdot
    \left(
    \frac{\partial g}{\partial u} \times \frac{\partial g}{\partial v}
    \right) \, du\,dv
\end{equation*}

\begin{enumerate}
    \item Parametrizar a superfície
\end{enumerate}

\subsubsection{Teorema de Divergência}

Seja $S$ a \textbf{superfície fechada}, orientada \textbf{positivamente} (p/
fora), fronteira de uma região sólida $E$.

\begin{equation*}
    \iint_S \vec{F} \cdot \vec{n} \, dS =
    \iiint_E \text{div} \vec{F} \, dV
\end{equation*}

\subsection{Teorema de Stokes}

Seja $S$ uma \textbf{superfície orientada}, cuja fronteira é formada por uma
curva $C$ fechada e com orientação positiva.

\begin{equation*}
    \oint_C \vec{F} \ dg =
    \iint_S \text{rot} \vec{F} \ dS =
    \iint_D \text{rot} \vec{F} \cdot \vec{N} \, du\,dv
\end{equation*}

\subsubsection{Trabalho}
\begin{equation*}
    W = \oint_C F \, dg =
    \int_{a}^{b} F\left(g(t)\right) \cdot g'(t) \, dt
\end{equation*}

\subsubsection{Potencial Vetorial}

Diz-se que $\vec{G}$ é potencial vetorial de $\vec{F}$ se $\vec{F} =$ rot\,$\vec{G}$

\begin{enumerate}
    \item Verificar que div\,$\vec{F} = 0$
    \item Resolver a equação $\vec{F} =$ rot\,$\vec{G}$:
          \begin{align*}
              \left(\vec{F_1},\,\vec{F_2},\,\vec{F_3}\right) & = \text{rot}\,\vec{G}
              \\ \Leftrightarrow
              \left(\vec{F_1},\,\vec{F_2},\,\vec{F_3}\right) & =
              \left(\frac{\partial G_3}{\partial y} - \frac{\partial G_2}{\partial z}
              ,\,\frac{\partial G_1}{\partial z} - \frac{\partial G_3}{\partial x}
              ,\,\frac{\partial G_2}{\partial x} - \frac{\partial G_1}{\partial y}
              \right)
          \end{align*}
    \item Assumir que uma das componentes de $G$ é nulo, (ex. $G_2$):
          \begin{equation*}
              \left(\vec{F_1},\,\vec{F_2},\,\vec{F_3}\right) =
              \left(\frac{\partial G_3}{\partial y}
              ,\,\frac{\partial G_1}{\partial z} - \frac{\partial G_3}{\partial x}
              ,\, - \frac{\partial G_1}{\partial y}
              \right)
          \end{equation*}
    \item Integrar as outras componente em ordem à variável corresponde ao
          componente anulado (ex. $y$):
          \begin{equation*}
              \int \vec{F_1} \,dy = \int \frac{\partial G_3}{\partial y} \,dy
          \end{equation*}
          \begin{equation*}
              \int \vec{F_3} \,dy = \int -\frac{\partial G_1}{\partial y} \,dy
          \end{equation*}
\end{enumerate}

\newpage

\section{Séries de Fourier}

Sendo $f(x)$ uma função $f$ com extensão periódica $\bar{f}$ com período $2L$:

\begin{equation*}
    SFf(x) = \frac{a_0}{2} + \sum_{n=1}^{+\infty}
    \left(
    a_n \cdot \cos\left(\frac{n\pi x}{L}\right) +
    b_n \cdot \sin\left(\frac{n\pi x}{L}\right)
    \right)
\end{equation*}

\begin{equation*}
    a_n =
    \frac{1}{L} \int_{-L}^{L} f(x) \cdot \cos\left(\frac{n\pi x}{L}\right) \, dx
    \ \ \ \Rightarrow \ \ \
    a_0 =
    \frac{1}{L} \int_{-L}^{L} f(x) \, dx
\end{equation*}

\begin{equation*}
    b_n =
    \frac{1}{L} \int_{-L}^{L} f(x) \cdot \sin\left(\frac{n\pi x}{L}\right) \, dx
\end{equation*}

\begin{center}
    \begin{tblr}[T]{colspec={ | Q[c,m] | Q[c,m] | Q[c,m] | }}
        \hline
        $\cos(x) = - \cos(x)$ (par)    &
        $\cos(n\pi) = (-1)^n$          &
        $\displaystyle \int_a^b \cos(*x) = \left[\frac{\sin(*x)}{*}\right]_a^b$ \\\hline
        $\sin(x) = - \sin(-x)$ (ímpar) &
        $\sin(n\pi) = 0$               &
        $\displaystyle \int_a^b \sin(*x) = \left[-\frac{\cos(*x)}{*}\right]_a^b$
        \\\hline
    \end{tblr}
\end{center}

\subsection{Teorema da Convergência Pontual}

\begin{equation*}
    SFf(x) =
    \begin{cases}
        f(x)                     & \text{se $x$ é um ponto de continuidade de $f$}    \\
        \frac{f(x^+)+f(x^-)}{2}  & \text{se $x$ é um ponto de descontinuidade de $f$} \\
        \frac{f(-L^+)+f(L^-)}{2} & \text{se $x = -L$ ou $x = L$}                      \\
    \end{cases}
\end{equation*}

\subsection{Série de Senos}

Quando a função $f(x)$ é \textbf{ímpar}:

\begin{equation*}
    S_{\sin} f(x) =
    \sum_{n=1}^{+\infty}b_n \cdot \sin\left(\frac{n\pi x}{L}\right)
\end{equation*}

\begin{equation*}
    b_n =
    \frac{2}{L} \int_{0}^{L} f(x) \cdot \sin\left(\frac{n\pi x}{L}\right) \, dx
\end{equation*}

\subsection{Série de Cossenos}

Quando a função $f(x)$ é \textbf{par}:

\begin{equation*}
    S_{\cos} f(x) = \frac{a_0}{2} + \sum_{n=1}^{+\infty}
    a_n \cdot \cos\left(\frac{n\pi x}{L}\right)
\end{equation*}

\begin{equation*}
    a_n =
    \frac{2}{L} \int_{0}^{L} f(x) \cdot \cos\left(\frac{n\pi x}{L}\right) \, dx
    \ \ \ \Rightarrow \ \ \
    a_0 =
    \frac{2}{L} \int_{0}^{L} f(x) \, dx
\end{equation*}

\subsubsection*{Integral por partes}

\begin{equation*}
    \int_a^b uv' = [uv] _a^b - \int_a^b u'v
\end{equation*}

\newpage

\section{Introdução às Equações Diferenciais Parciais}

\subsection{Método de Separação de Variáveis}

\begin{equation*}
    u(t, x) = T(t)X(x)
\end{equation*}
\begin{equation*}
    \text{Condição de fronteira (CF):\ \ } u(t, 0) = u(t, \pi) = 0 \text{\ \ ($t$ variável)}
\end{equation*}
\begin{equation*}
    \text{Condição inicial (CI):\ \ } u(0, x) = f(x) \text{\ \ ($t$ fixo $= 0$)}
\end{equation*}

\begin{enumerate}
    \item Substituir $u(t, x)$ na equação diferencial por $T(t)X(x)$
    \item Separar as variáveis dos dois lados da igualdade:
          \begin{equation*}
              \frac{T'}{T} = \frac{X''}{X}
          \end{equation*}
    \item Igualar os dois lados da equação a lambda ($\lambda$):
          \begin{equation*}
              \frac{T'}{T} = \lambda \text{\ \ \ \ e \ \ \ }
              \frac{X''}{X} = \lambda
          \end{equation*}
    \item Analisar as condições de fronteira, sabendo que $T(t) \neq 0$:
          \begin{equation*}
              u(t, 0) = u(t, \pi) = 0 \ \Rightarrow\ X(0) = X(\pi) = 0
          \end{equation*}
    \item Construir os dois problemas (EDOs)a resolver:
          \begin{equation*}
              \text{P1:\ \ \ }
              \begin{cases}
                  X'' - \lambda X = 0 \\
                  X(0) = X(\pi) = 0
              \end{cases}
              \text{se $x \in\ ]0, \pi[$}
          \end{equation*}
          \begin{equation*}
              \text{P2:\ \ \ } \frac{T'}{T} = \lambda \ \Leftrightarrow \
              T' = \lambda T \ \Leftrightarrow \
              T' - \lambda T = 0 \ \Rightarrow
              P(R) = R-\lambda = 0
          \end{equation*}
    \item Resolver P1 testando as várias possibilidades para $\lambda$:
          \begin{itemize}
              \item $\lambda = 0$
              \item $\lambda > 0$
              \item $\lambda < 0$
          \end{itemize}
    \item Resolver P2 para os valores de $\lambda$ obtidos anteriormente
    \item Combinar os resultados obtidos, obtendo a solução do PVF:
          \begin{equation*}
              u_k(t, x) = T_k(t) X_k(t) \ \Rightarrow \
              u(t,x) = \sum_{k=1}^{\infty} c_k T_k X_k
          \end{equation*}
    \item Calcular as constantes $c_k$ utilizando a condição inicial:
          \begin{equation*}
              u(0,x) = \sum_{k=1}^{\infty} c_k X_k = \text{CI}
          \end{equation*}
    \item Substituir os valores obtidos em $u(t,x)$
\end{enumerate}

\newpage

\section{Extras}

\subsection{Mudança de Variáveis de Integração}

\subsubsection{Coordenadas Polares}

\begin{equation*}
    \iint f(r\cos\theta, r\sin\theta) \cdot r\,dr\,d\theta
\end{equation*}

\subsubsection{Coordenadas Cilíndricas}

\begin{equation*}
    \iiint f(r\cos\theta, r\sin\theta, z) \cdot r\,dz\,dr\,d\theta
\end{equation*}

\subsubsection{Coordenadas Esféricas}

\begin{equation*}
    \iiint f(r\sin\varphi\cos\theta, r\sin\varphi\sin\theta, r\cos\varphi) \cdot r^2\sin\varphi\,dr\,d\varphi\,d\theta
\end{equation*}

\end{document}