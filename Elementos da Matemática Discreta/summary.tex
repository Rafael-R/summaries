\documentclass[11pt, a4paper]{article}
\usepackage[margin=2cm]{geometry}
\usepackage{hyperref}
\usepackage{enumitem}
\usepackage{amsmath,amsfonts,amssymb, amstext}
\usepackage{array}
\usepackage{cancel}

\newcolumntype{C}{>{$}c<{$}}

\hypersetup{
    colorlinks,
    citecolor=black,
    filecolor=black,
    linkcolor=black,
    urlcolor=black
}

\setlength{\parindent}{0pt}

\begin{document}

\begin{titlepage}
    \begin{center}
        \vspace*{1cm}

        \textbf{\LARGE Elementos de Matemática Discreta}
        \vspace{0.5cm}

        \Large Resumo
        \vspace{1.5cm}

        \textbf{Rafael Rodrigues}
        \vfill
        LEIC \\
        Instituto Superior Técnico \\
        2023/2024
    \end{center}
\end{titlepage}

\tableofcontents

\newpage

\section{Teste 1}


\section{Teste 2}

\subsection*{Congruências}

\begin{enumerate}
    \item $a + c \equiv_n b + d$
    \item $a^k \equiv_n b^k \rightarrow a \equiv_n b$
    \item $-$
    \item $-$
    \item $\displaystyle ac \equiv_n bc \rightarrow
              a \equiv_{\frac{n}{n \frown c}} b$
    \item $x \equiv_{pq} a$ , sse  $x \equiv_p a$ e $x \equiv_q a$
          com $p \frown q = 1$
\end{enumerate}

\subsection{Teorema Chinês do Resto}

\begin{enumerate}
    \item Escrever o sistema na forma normal
          \begin{equation*}
              \begin{cases}
                  \ x \equiv_{c_1} a_1 \\
                  \ x \equiv_{c_2} a_2 \\
                  \ x \equiv_{c_3} a_3
              \end{cases}
          \end{equation*}
    \item Verificar se os módulos são primos entre si
          \begin{enumerate}
              \item caso sejam:
                    \begin{enumerate}
                        \item $M = \prod c_i$
                    \end{enumerate}
              \item caso não sejam:
                    \begin{enumerate}
                        \item Verificar se $a_i - a_j = xt$ para $x = gcd(c_i, c_j)$
                        \item $M = lcm(c_1, ..., c_k)$
                        \item Fatorizar $M$ e utilizar cada um dos fatores como novo $c_i$
                    \end{enumerate}
          \end{enumerate}
    \item Aplicar o algoritmo
          \begin{enumerate}
              \item Construir a tabela:
                    \begin{center}
                        \begin{tabular}{ | C | C | C | C | C | C | }
                            \hline
                            a_i         & c_i                 & n_i            & mod(n_i, c_i) &
                            \tilde{n}_i & a_i n_i \tilde{n}_i                                        \\\hline
                                        &                     & c_2 \times c_3 &               &   & \\\hline
                                        &                     & c_1 \times c_3 &               &   & \\\hline
                                        &                     & c_1 \times c_2 &               &   & \\\hline
                        \end{tabular}
                    \end{center}
              \item Calcular a solução particular: $x_0 = \sum a_i n_i \tilde{n}_i$
              \item Calcular a solução geral: $x = x_0 + Mk$, com $k \in \mathbb{Z}$.
              \item Escolher a solução que se enquadra no problema.
          \end{enumerate}
\end{enumerate}

\newpage
\subsection{Algoritmo RSA}

\begin{enumerate}
    \item Criar as chaves, usando dois números primos diferentes $p$ e $q$:
          \begin{itemize}
              \item Chave pública $(N, e)$
              \item Chave privada $(N, d)$
          \end{itemize}
          \begin{enumerate}
              \item $N = p \times q$
              \item Para encontrar $e$ ou $d$ resolver equação diofantina:
                    $1 = (e \times d) + \left[(p - 1)(q - 1) \times k\right]$

          \end{enumerate}
    \item Encriptar mensagem $M$ usando a chave pública: $M^e \equiv_N R$
    \item Desencriptar mensagem $R$ usando a chave privada: $R^d \equiv_N M$
\end{enumerate}

\subsection*{Algoritmo de Saunderson}

\begin{minipage}{0.49\textwidth}
    \begin{center}
        \begin{tabular}{ C | C | C | C }
            a_i & q_i & x_i & y_i \\\hline
            100 &     & 1   & 0   \\\hline
            49  & 2   & 0   & 1   \\\hline
            2   & 24  & 1   & -2  \\\hline
            1   & 2   & -24 & 49  \\\hline
            0   &     &     &
        \end{tabular}
    \end{center}
\end{minipage}
\begin{minipage}{0.49\textwidth}
    \begin{flalign*}
        x_{i+1} = x_{i-1} - q_i x_i
    \end{flalign*}
    \begin{flalign*}
        y_{i+1} = y_{i-1} - q_i y_i
    \end{flalign*}
\end{minipage}

\subsection{Equações Diofantinas}

\begin{enumerate}
    \item Escrever equação diofantina: $(\ )x \pm (\ )y = $
\end{enumerate}


\subsection{Pequeno Teorema de Fermat}


\section{Teste 3}

\subsection{Funções Geradoras}

\subsection{Algoritmo FFT}

\subsection{Grafos}

\subsubsection{Algoritmo de Kruskal}

\subsubsection{Algoritmo de Dijkstra}

\subsubsection{Teorema de Kuratowski}


\end{document}