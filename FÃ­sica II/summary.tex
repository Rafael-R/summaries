\documentclass[11pt]{article}
\usepackage[paper=letterpaper,margin=2cm]{geometry}
\usepackage{enumitem}
\usepackage{amsmath}
\usepackage{amsfonts}
\usepackage{tabularray}
\usepackage{mathtools}

\setlength{\parindent}{0pt}

\newcommand{\twopartdef}[4]
{ \displaystyle
	\left\{
		\begin{array}{ll}
			#1, & \mbox{se } #2 \\
			#3, & \mbox{se } #4
		\end{array}
	\right.
}

\begin{document}

\begin{titlepage}
    \begin{center}
        \vspace*{1cm}

        \textbf{\LARGE Física II}
        \vspace{0.5cm}

        \Large Resumo
        \vspace{1.5cm}

        \textbf{Rafael Rodrigues}
        \vfill
        LEIC \\
        Instituto Superior Técnico \\
        2023/2024
    \end{center}
\end{titlepage}

\tableofcontents

\newpage

\section{Fundamentos}

\section{Eletrostática}

\section{Condutores e Dielétricos}

\end{document}