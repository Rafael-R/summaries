\documentclass[11pt]{article}
\usepackage[paper=letterpaper,margin=2cm]{geometry}
\usepackage{enumitem}
\usepackage{amsmath}
\usepackage{amsfonts}
\usepackage{tabularray}
\usepackage{mathtools}
\usepackage{hyperref}

\hypersetup{
    colorlinks,
    citecolor=black,
    filecolor=black,
    linkcolor=black,
    urlcolor=black
}

\setlength{\parindent}{0pt}

\begin{document}

\begin{titlepage}
    \begin{center}
        \vspace*{1cm}

        \textbf{\LARGE Física II}
        \vspace{0.5cm}

        \Large Resumo
        \vspace{1.5cm}

        \textbf{Rafael Rodrigues}
        \vfill
        LEIC \\
        Instituto Superior Técnico \\
        2023/2024
    \end{center}
\end{titlepage}

\tableofcontents

\newpage

\section{Fundamentos}

\subsubsection*{Campos}

Um campo $(\phi)$ é uma zona do espaço onde em cada ponto está definida uma quantidade. 

\section{Eletrostática}

\section{Condutores e Dielétricos}

\subsection{Condutor}

Num condutor as cargas elétricas podem mover-se livremente no material. Um condutor pode ser metálico onde as cargas são eletrões, ou líquido onde as cargas são iões.

\subsubsection{Propriedades}

\begin{itemize}
    \item $\overrightarrow{E} = 0$ dentro de um condutor\\[6pt]
    Quando aplicamos um campo $\overrightarrow{E}$ sobre um condutor \textbf{isolado}, as cargas negativas (\textbf{cargas induzidas}) movem-se na direção oposta ao campo $\overrightarrow{E}$, separando-se das cargas positivas e criando um campo $\overrightarrow{E'}$.
    \[ \overrightarrow{E} + \overrightarrow{E'} = 0 = \overrightarrow{E}_{total} \]
    \item $\rho = 0$ dentro de um condutor\\[6pt]
    Pela lei de Gauss $\displaystyle \overrightarrow{\nabla}.\overrightarrow{E}=\frac{\rho}{\epsilon_0}$, como $\overrightarrow{E} = 0 \Longrightarrow \rho = 0$.\\
    Há cargas no interior do condutor, mas em igual quantidade de + e -, de modo a que se anulam.
    \item $\overrightarrow{E} \perp$ à superfície do condutor\\[6pt]
    Qualquer carga remanescente situa-se na superfície do condutor, o que significa que 
    \item um condutor é um equipotencial
\end{itemize}

\subsubsection{Cargas Induzidas}

As Cargas procuram sempre o equilíbrio ($\overrightarrow{E}=0$).\\[2pt]
Se tivermos uma carga $+\mathbf{Q}$ e um condutor não carregado, as cargas negativas são atraídas para a carga $+\mathbf{Q}$, para que assim se anule o campo no interior do condutor.

\subsubsection{Cavidades}

No caso de no interior do condutor houver uma cavidade e a carga $+\mathbf{Q}$ estiver dentro dela, então o campo é não nulo nessa região.\\[2pt]
Assim a carga induzida $\mathbf{Q}_{ind}$ é igual a $-\mathbf{Q}$ e a carga à superfície do condutor passa a ser positiva porque as cargas negativas aproximaram-se da carga $+\mathbf{Q}$ deixando de estar na superfície.\\[2pt]
No caso de não haver carga na cavidade, $\overrightarrow{E}=0$ na cavidade (Gaiola de Faraday).

\break

\subsection{Condensador}

Um condensador é um componente que armazena cargas elétricas num campo elétrico, composto por dois condutores, com cargas $+\mathbf{Q}$ e $-\mathbf{Q}$.

\subsubsection{Condensador de placas paralelas}

Composto por duas placas paralelas a uma distância $d$.

\begin{itemize}
    \item Densidade de carga de cada placa: $\displaystyle \sigma = \frac{Q}{A}$
    \item Campo elétrico entre as placas: $\displaystyle \overrightarrow{E} = \frac{Q}{A\epsilon_0}$
    \item Diferença de potencial entre as placas: $\displaystyle V = \frac{Qd}{A\epsilon_0}$
    \item Capacitância: $\displaystyle C = \frac{\mathbf{A}\epsilon_0}{d}$
\end{itemize}

\subsubsection{Condensador de esferas concêntricas}

Composto por duas superfícies esféricas concêntricas de raios $a$ e $b$.

\begin{itemize}
    \item campo entre as superfícies: $\displaystyle \overrightarrow{E} = \frac{1}{4\pi\epsilon_0}\frac{\mathbf{Q}}{r^2}\overrightarrow{e}_r$
    \item Diferença de potencial entre as superfícies: $\displaystyle V = \frac{\mathbf{Q}}{4\pi\epsilon_0}\left(\frac{1}{a}-\frac{1}{b}\right)$
    \item Capacitância: $\displaystyle C = 4\pi\epsilon_0\frac{ab}{b-a}$
\end{itemize}

\subsubsection{Condensadores em Paralelo}

\[ C = \frac{Q_1+Q_2+Q_3}{V} 
     = \frac{Q_1}{V}+\frac{Q_2}{V}+\frac{Q_3}{V} 
     = C_1+C_2+C_3 \]

\subsubsection{Condensadores em Série}

\[ C = \frac{Q}{V_1+V_2+V_3} \Rightarrow \frac{1}{C}
     = \frac{1}{C_1}+\frac{1}{C_2}+\frac{1}{C_3} \]

\subsubsection{Trabalho}
Para carregar um condensador é preciso eliminar eletrões do condutor positivo e movê-los para o condensador negativo. Isso requer trabalho pois é temos de puxar cargas negativas contra o campo elétrico.\\[2pt]
O trabalho necessário para carregar o condensador com uma carga $mathbf{Q}$ é dado por:

\[ W = \frac{Q^2}{2C} = \frac{CV^2}{2} \]

\subsection{Dielétricos}

Com os Dielétricos entramos no estudo do campo elétrico na matéria. Existem 2 grandes grupos:

\begin{itemize}
    \item Condutores
    \begin{itemize}
        \item As cargas elétricas movem-se livremente através do material
    \end{itemize}
    \item Dielétricos ou Isolantes: 
    \begin{itemize}
        \item As cargas elétricas estão presas aos átomos ou moléculas e apenas se podem mover um pouco dentro deles
        \item Existem 2 mecanismos pelos quais um campo elétrico pode distorcer a distribuição de carga de um átomo ou molécula dielétrica
        \begin{itemize}
            \item Estiramento
            \item Rotação
        \end{itemize}
    \end{itemize}
\end{itemize}

\section{Magnetostática}

\subsection{Corrente Elétrica}

\subsubsection{Densidade de Corrente}

\subsubsection{Corrente Elétrica}

\subsubsection{Lei de Kirchhoff}

\subsection{Lei de Ohm}

Para ter corrente é preciso empurrar as cargas, a velocidade que adquirem depende das características do material. A densidade de corrente elétrica $\overrightarrow{J}$ é proporcional à força aplicada
\[ \overrightarrow{J} = \sigma \overrightarrow{f}\]
em que $\sigma$ é uma constante experimental que depende do material e que se chama \textbf{Condutividade} do meio. É mais comum o uso da \textbf{Resistividade} do meio, que é o inverso: $\displaystyle \rho = \frac{1}{\sigma}$\\

Nesta parte da matéria assumimos que a força aplicada às cargas é a do Campo Elétrico
\[ \overrightarrow{J} = \sigma \overrightarrow{E}\]

Num condutor \textbf{em equilíbrio eletrostático} temos $\overrightarrow{E}=0$ e $\overrightarrow{J}=0$.\\
Para condutores perfeitos temos $\displaystyle \overrightarrow{E}=\frac{\overrightarrow{J}}{\sigma}=0$, mesmo que esteja corrente a fluir.\\

Conclui-se que o campo elétrico necessário para movimentar as cargas é quase nulo. Assim consideramos estes fios como \textbf{equipotenciais}. Já as resistências são feitos de materiais que conduzem pouco.\\

A Lei de Ohm é então dada por

\[ \displaystyle R=\frac{V}{I} \Leftrightarrow V=I\times R \]
\begin{center}
    $V$: Tensão elétrica (V),
    $R$: Resistência elétrica ($\Omega$), 
    $I$: Corrente elétrica (A)
\end{center}

\[ \displaystyle R = \frac{\rho L}{A} \Leftrightarrow R=\frac{L}{A\sigma} \]
\begin{center}
    $R$: Resistência elétrica ($\Omega$),
    $L$: Comprimento do condutor (m),
    $A$: Secção do condutor (m$^2$)
\end{center}

\[ \displaystyle I = \sigma EA \]
\begin{center}
    $I$: Corrente elétrica (A),
    $E$: Campo elétrico (Vm$^{-1}$),
    $A$: Secção do condutor (m$^2$)
\end{center}

\subsection{Lei de Joule}

\[ P=VI=RI^2 \]
\begin{center}
    $P$: Potência elétrica (W),
    $V$: Tensão elétrica (V),
    $I$: Corrente elétrica (A),
    $R$: Resistência elétrica ($\Omega$)
\end{center}

\subsection{A Força de Lorentz}

\subsection{A Lei de Biot-Savart}

O campo magnético gerado por um fio percorrido por corrente estacionária é dada por
\[ \displaystyle \overrightarrow{B}(\overrightarrow{r}) = \frac{\mu_0\mathbf{I}}{4\pi} \int_{\mathcal{C}}\frac{d\overrightarrow{\mathbf{I}}'\times \overrightarrow{e}_r}{r^2} \]

A qunatidade $\mu_0$ chama-se \textbf{permeabilidade do espaço livre} com valor $\mu_0 = 4\pi \times 10^{-7} N A^{-2}$.

\subsubsection{Experiência de Ampère revisitada}

Para calcular o campo magnético de um fio longo sobre outro fio longo, paralelos entre si, a uma distância $d$, percorridos por correntes $\mathbf{I}_1$ e $\mathbf{I}_2$ usamos a fórmula
\[ \displaystyle \overrightarrow{B}_1 = -\frac{\mu_0\mathbf{I}_1}{2\pi d} \]

Para calcular o valor da força por unidade de cumprimento dentre os dois fios utilizamos a fórmula
\[ \displaystyle \overrightarrow{F} = \frac{\mu_0\mathbf{I}_1\mathbf{I}_2}{2\pi d} \]

Esta força é atrativa caso as correntes tenham a mesma direção, ou repulsiva caso contrário.

\section{Campo magnético na matéria}

\section{Eletrodinâmica}

\section{As equações de Maxwell}

\end{document}