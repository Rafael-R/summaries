\documentclass[11pt]{article}
\usepackage[paper=letterpaper,margin=2cm]{geometry}
\usepackage{enumitem}
\usepackage{amsmath}
\usepackage{amsfonts}
\usepackage{tabularray}
\usepackage{mathtools}
\usepackage{hyperref}

\hypersetup{
    colorlinks,
    citecolor=black,
    filecolor=black,
    linkcolor=black,
    urlcolor=black
}

\setlength\parskip{1em plus 0.1em minus 0.2em}
\setlength\parindent{0pt}

\begin{document}

\begin{titlepage}
    \begin{center}
        \vspace*{1cm}

        \textbf{\LARGE Gestão}
        \vspace{0.5cm}

        \Large Resumo
        \vspace{1.5cm}

        \textbf{Rafael Rodrigues}
        \vfill
        LEIC \\
        Instituto Superior Técnico \\
        2023/2024
    \end{center}
\end{titlepage}

\tableofcontents

\newpage

\section{Introdução}

\subsection{Conceitos fundamentais sobre Gestão}

\textbf{Gestão -} processo que tem como função atingir as metas e objetivos de uma organização, de forma eficiente e eficaz.

\textbf{Eficiência -} atingir um objetivo com o mínimo de recursos.

\textbf{Eficácia -} atingir o objetivo.

\textbf{Organização -} entidade social direcionada por objetivos e (deliberadamente) estruturada.

\subsection{Funções da Gestão}

O processo de gestão divide-se em 4 funções:
\begin{itemize}[topsep=0pt]
    \item \textbf{Planeamento -} Definir objetivos e ações. Decidir que tarefas serão executadas e alocar recursos para as mesmas e calendarizá-las.
    \item \textbf{Organização -} Designa e agrupa tarefas, alocando-as às várias estruturas da organização.
    \item \textbf{Liderança -} Utilizar influência para gerir o estado emocional do corpo trabalhador para atingir os objetivos da organização.
    \item \textbf{Controlo -} Monitorizar atividades do corpo trabalhador, determinar se a organização se está a aproximar dos objetivos e fazer correções quando necessário.
\end{itemize}

\subsection{Competências de um Gestor}

\begin{itemize}
    \item \textbf{Conceptuais -} capacidade de ver (e agir na) a organização como um todo (componentes e relações entre elas) e estabelecer as relações com o exterior. Requer a capacidade de pensar estrategicamente. Fundamental na gestão de topo.
    \item \textbf{Humanas -} capacidade em trabalhar com outras pessoas quer individualmente quer em equipa. Fundamental na gestão de topo e intermédia.
    \item \textbf{Técnicas -} domínio de tarefas específicas (métodos, técnicas, outros conhecimentos). Têm menor peso na gestão de topo.
\end{itemize}

\subsection{Ética}

\subsection{Responsabilidade Social}

\subsection{Estrutura Organizacional}

Conjunto de tarefas formais atribuídas a entidades da empresa (indivíduos, equipas e departamentos) conjugados com as linhas de autoridade, ónus de decisão e hierarquias
incluindo ainda sistemas de coordenação e controlo.

\subsubsection{Tipos de Estrutura}

\begin{itemize}
    \item Funcional
    \item Divisional
    \item Matricial
    \item Em Equipa
    \item Rede
\end{itemize}

\begin{tabular}[t]{ | c | p{100pt} | p{150pt} | }
    \hline
    Tipo de Estrutura & Vantagens                              & Desvantagens                                             \\\hline
    Funcional         & especialização, economias de escala    & fraca comunicação entre departamentos                    \\\hline
    Divisional        & flexibilidade, adaptação               & duplicação de recursos, fraca coordenação entre divisões \\\hline
    Matricial         & interdisciplinaridade                  & conflitos na cadeia de comando                           \\\hline
    Em Equipa         & rapidez de resposta, entusiasmo        & tempo em reuniões, conflitos eventuais                   \\\hline
    Rede              & flexibilidade, poucos custos estrutura & coordenação e controlo mais difícil                      \\\hline
\end{tabular}

\newpage

\section{Informação Financeira}

\textbf{Contabilidade -} Processo formal de identificar, medir e comunicar informação sobre o património e resultados de uma empresa para os níveis de decisão internos ou agentes externos.

\subsection{Organização da Informação Financeira}

\begin{itemize}
    \item Contabilidade Geral (Financeira ou Externa)
          \begin{itemize}
              \item Gera informação para os elementos externos à empresa         (reguladores, fornecedores, acionistas, bancos, etc.).
              \item Segue as normas internacionais de contabilidade em vigor.
              \item O Sistema de Normalização Contabilística assimila a
                    transposição das diretivas contabilísticas da União Europeia.
          \end{itemize}
    \item Contabilidade Analítica (de Gestão ou Interna)
          \begin{itemize}
              \item Gera informação específica e desagregada para apoiar a gestão.
              \item Apura resultados por produtos, regiões, mercados, atividades, etc.
              \item É a base para a orçamentação e análise de custos.
          \end{itemize}
\end{itemize}

\subsection{Principais Mapas Contabilísticos}

\subsubsection{Balanço}

O balanço permite registar as contas de um agente económico. É composto por 3 grandes rubricas, que depois podem ser distinguidas em subpartes:

\begin{itemize}
    \item \textbf{Ativo -} Bens e direitos que a empresa possui ou tem direito a receber.
          \begin{itemize}
              \item \textbf{Ativo não Corrente -} Ativo que por natureza, não é volátil.
                    \begin{itemize}
                        \item \textbf{Ativos Fixos Tangíveis} (Edifícios, equipamentos, ...)
                        \item \textbf{Ativos Intangíveis} (Marcas, patentes, ...)
                    \end{itemize}
              \item \textbf{Ativo Corrente -} Ativos voláteis.
                    \begin{itemize}
                        \item \textbf{Inventários} (Produtos fabricados, em vias de fabricação ou matéria prima)
                        \item \textbf{Valores Monetários} (Dinheiro, depósitos e títulos financeiros)
                        \item \textbf{Dívidas de clientes}
                    \end{itemize}
          \end{itemize}
    \item \textbf{Passivo -} Soma das dívidas (responsabilidades) de um agente económico.
          \begin{itemize}
              \item \textbf{Passivo não Corrente -} Financiamentos e outras dívidas (a pagar em mais de 1 ano)
              \item \textbf{Passivo Corrente -} Dívida a fornecedores, estado, financiamentos ou outras (a pagar em menos de 1 ano)
          \end{itemize}
    \item \textbf{Capital Próprio -} Capital realizado e lucros do período ou de períodos anteriores retidos na empresa.
\end{itemize}

\newpage

\subsubsection*{Equação Fundamental da Contabilidade}

\begin{center}
    \textbf{Ativo = Passivo + Capital Próprio}
\end{center}

Se Ativo $>$ Passivo $\implies $ Capital Próprio $>$ 0 \\
Se Ativo $<$ Passivo $\implies $ Capital Próprio $<$ 0 (falência técnica)


\subsubsection{Demonstração dos Fluxos de Caixa}

\textbf{Ótica da Caixa -} Permite ver o dinheiro que uma empresa tem num determinado momento, a Liquidez.

\begin{center}
    Saldo inicial + Dinheiro recebido + Pagamentos = Saldo Final
\end{center}

\subsubsection{Demonstração de Resultados}

\textbf{Ótica de Exercício -} Permite ver se a empresa é rentável.

\begin{center}
    Rendimentos - Gastos = Resultado Líquido do Período
\end{center}

\begin{center}
    Demonstração de Resultados \\
    \begin{tabular}[t]{ | p{250pt} | }
        \hline Valor das vendas                                                   \\
        \hline Custo das vendas                                                   \\
        \hline \multicolumn{1}{ |r| }{\textbf{Resultado Bruto (RB)}}              \\
        \hline Outros Rendimentos                                                 \\
        \hline Outros Gastos                                                      \\
        \hline \multicolumn{1}{ |r| }{\textbf{Resultado Operacional (RO)}}        \\
        \hline Juros                                                              \\
        \hline \multicolumn{1}{ |r| }{\textbf{Resultado Antes de Impostos (RAI)}} \\
        \hline Impostos                                                           \\
        \hline \multicolumn{1}{ |r| }{\textbf{Resultado Líquido (RL)}}            \\
        \hline
    \end{tabular}
\end{center}

\subsection{Análise de Rácios Financeiros}

\textbf{Rácios -} Indicadores de gestão que exprimem uma relação entre elementos dos documentos contabilísticos (Balanço, Demonstração de Resultados) e a partir dos quais é possível tirar ilações sobre a situação da empresa (Solidez Financeira e níveis de desempenho económico e financeiro).

Dizemos que uma empresa tem \textbf{Solidez Financeira}:
\begin{itemize}
    \item Quanto maior o capital próprio e menor o passivo (melhor ainda se o passivo for não corrente).
    \item Quanto maior for o somatório do capital próprio com o passivo não corrente, relativamente ao ativo corrente.
    \item Quanto maior a rentabilidade do capital total em relação ao juro a pagar pelo capital alheio.
\end{itemize}

\subsubsection{Rácios de Rentabilidade}

Indicam a rentabilidade do capital próprio, ativo ou vendas.

\begin{equation*}
    \textrm{Rentabilidade do Capital Próprio (RCP)} =
    \frac{\textrm{Resultado Líquido}}{\textrm{Capital Próprio}}
\end{equation*}

\begin{equation*}
    \textrm{Rentabilidade das Vendas} =
    \frac{\textrm{Resultado Operacional}}{\textrm{Vendas}}
\end{equation*}

\subsubsection{Rácios de Atividade ou Funcionamento}

Indicam o grau de utilização dos recursos da empresa.

\begin{equation*}
    \textrm{Prazo Médio de Recebimentos (em dias)} =
    \frac{\textrm{Clientes}}{\textrm{Vendas}} \times 365
\end{equation*}

\begin{equation*}
    \textrm{Prazo Médio de Pagamentos (em dias)} =
    \frac{\textrm{Fornecedores}}{\textrm{Compras}} \times 365
\end{equation*}

\begin{equation*}
    \textrm{Rotação de Inventários} =
    \frac{\textrm{Custo das Vendas}}{\textrm{Inventários médios}}
\end{equation*}

\subsubsection{Rácios de Solvabilidade}

Indicam a capacidade da empresa de satisfazer os compromissos financeiros de médio e longo prazo.

\begin{equation*}
    \textrm{Solvabilidade Total (Autonomia Financeira)} =
    \frac{\textrm{Capital Próprio}}{\textrm{Ativo}}
\end{equation*}

Uma boa solvabilidade total corresponde a valores acima de 1/3.

\begin{equation*}
    \textrm{Solvabilidade Reduzida} =
    \frac{\textrm{Capital Próprio}}{\textrm{Passivo}}
\end{equation*}

Uma boa solvabilidade reduzida corresponde a valores acima de 1/2.

\subsubsection{Rácios de Liquidez}

Indicam a capacidade de a empresa satisfazer os compromissos financeiros de curto prazo através do fundo de maneio.

\textbf{Fundo de maneio -} diferenca entre Ativo Corrente e Passivo Corrente.

\begin{equation*}
    \textrm{Liquidez Geral} =
    \frac{\textrm{Ativo Corrente}}{\textrm{Passivo Corrente}} =
    \frac{\textrm{Caixa + Clientes + Inventário}}{\textrm{Passivo Corrente}}
\end{equation*}

\begin{equation*}
    \textrm{Liquidez Reduzida} =
    \frac{\textrm{Ativo Corrente - Inventário}}{\textrm{Passivo Corrente}} =
    \frac{\textrm{Caixa + Clientes}}{\textrm{Passivo Corrente}}
\end{equation*}

\subsection{Análise Custo-Volume-Resultado}

\textbf{Custos Fixos -} Gastos em que a empresa incorre independentemente da quantidade produzida (ex: Gastos de instalação).

\textbf{Custos Variáveis -} Variam proporcionalmente com a quantidade produzida (ex: custos de matéria-prima).

\textbf{Ponto Crítico -} Nível de atividade que corresponde a Lucro zero, ou seja, a quantidade produzida a partir do qual a empresa passa a ter lucro (ou seja, passa a ser rentável).
\begin{align*}
    Lucro = 0 & \implies p * Q - CV - CF = 0       \\
              & \implies p * Q - cv_u * Q - CF = 0 \\
              & \implies (p - cv_u) * Q - CF = 0   \\
              & \implies mc_u * Q = CF
\end{align*}
\begin{center}
    \begin{tabular}{l}
        $Q$ - Quantidade produzida e vendida         \\
        $p$ - Preço de venda unitário                \\
        $CV$ - Total dos Custos Variáveis            \\
        $CF$ - Total dos Custos Fixos                \\
        $cv_u$ - Custo variável unitário (constante) \\
    \end{tabular}
\end{center}

\begin{tabular}{ c c }
    \begin{minipage}{0.45\textwidth}
        \begin{equation*}
            mc_u = p - cv_u
        \end{equation*}
    \end{minipage} &
    \begin{minipage}{0.45\textwidth}
        $mc_u$ - Margem de contribuição unitária
    \end{minipage}
\end{tabular}

\begin{tabular}{ c c }
    \begin{minipage}{0.45\textwidth}
        \begin{equation*}
            Q_c = \frac{CF}{mc_u}
        \end{equation*}
    \end{minipage} &
    \begin{minipage}{0.45\textwidth}
        $Q_c$ - Quantidade crítica a partir da qual há lucro
    \end{minipage}
\end{tabular}

\begin{tabular}{ c c }
    \begin{minipage}{0.45\textwidth}
        \begin{equation*}
            R_c = p \times Q_c
        \end{equation*}
    \end{minipage} &
    \begin{minipage}{0.45\textwidth}
        $R_c$ - Receita crítica a partir da qual há lucro
    \end{minipage}
\end{tabular}

\newpage

\section{Análise de Projetos de Investimento}

\textbf{Investimento -} Aplicação de recursos, com vista em obter retorno futuro.

\subsection{Como calcular valores atuais e futuros}

\textbf{Juro -}
\begin{itemize}[topsep=0pt]
    \item \textbf{Juro Simples -}
          \begin{equation*}
              rC_0+C_0=(1+r)C_0
          \end{equation*}
    \item \textbf{Juro Composto -}
          \begin{itemize}
              \item \textbf{Capitalização -}
                    \begin{equation*}
                        C_n=C_i(1+r)^{n-i}
                    \end{equation*}
              \item \textbf{Atualização -}
                    \begin{equation*}
                        C_i=\frac{C_n}{(1+r)^{n-1}}
                    \end{equation*}
          \end{itemize}
\end{itemize}

\textbf{Taxa de Juro Nominal ($r_n$) -} usa-se na avaliação de projetos a preços correntes, não está ajustada à inflação.

\textbf{Taxa de Juro Real ($r_r$) -} usa-se na avaliação de projetos a preços constantes, ajustada à inflação ($i$).
\begin{equation*}
    r_r = \frac{1+r_n}{1+i} - 1 \approx r_n - i
\end{equation*}

\textbf{Taxas Anuais Nominais (TAN) -}

\textbf{Taxas Anuais Efetivas (TAE) -}

Taxa de Inflação

\textbf{Anuidade -} Pagamento que é feito repetidamente, durante $n$ períodos. Seja $r$ a taxa de atualização, ou seja, o valor pelo qual se atualiza o pagamento, a cada iteração. Seja $A$ o valor da anuidade.
\begin{itemize}[topsep=0pt]
    \item Valor do $i$-gésimo pagamento: $\displaystyle p_i=A\frac{1}{(1+r)^i}$
    \item Valor atual (soma das anuidades): $\displaystyle VA=A\frac{(1+r)^n-1}{(1+r)^n\times r}=Af(r,n)$\\[6pt]
          A $f(r,n)$ chamamos \textbf{fator de anuidade}. Quando $\displaystyle n = \infty \Rightarrow f(r,\infty)=\frac{1}{r}$
\end{itemize}

\subsection{Análise da Rentabilidade de Projetos de Investimento}

Cash Flow (VR, VAL, TIR, PRI, IR), Custo Médio Ponderado do Capital

\newpage

\section{Gestão Estratégica}

\section{Marketing}

\end{document}