\documentclass[12pt]{article}
\usepackage[paper=letterpaper,margin=2cm]{geometry}
\usepackage{enumitem}
\usepackage{amsmath}
\usepackage{amsfonts}
\usepackage{tabularray}
\usepackage{mathtools}

\setlength{\parindent}{0pt}


\begin{document}

\begin{titlepage}
    \begin{center}
        \vspace*{1cm}

        \textbf{\LARGE Inteligência Artificial}
        \vspace{0.5cm}

        \Large Resumo
        \vspace{1.5cm}

        \textbf{Rafael Rodrigues}
        \vfill
        LEIC \\
        Instituto Superior Técnico \\
        2022/2023
    \end{center}
\end{titlepage}

\tableofcontents

\newpage

\section{Introdução}



\newpage

\section{Agentes Inteligentes}



\newpage

\section{Resolução de Problemas com Procura}



\subsection{Procura não Informada}



\subsubsection{Breadth-First Search}



\subsubsection{Uniform-Cost Search}



\subsubsection{Depth-First Search}



\subsubsection{Depth-Limited Search}



\subsubsection{Iterative Deepening Search}



\subsubsection{Bidirectional Search}



\subsubsection*{Resumo dos Algoritmos}



\newpage

\subsection{Procura Informada}



\subsubsection{Greedy Best-First Search}



\subsubsection{A* Search}



\subsubsection{Iterative Deepening A* Search}



\subsubsection{Recursive Best-First Search}


\newpage

\subsection{Funções Heurísticas}



\newpage

\section{Procura em Ambientes Complexos}



\subsection{Procura local}



\subsection{Procura em Espaços Contínuos}



\subsection{Procura com Ações não Determinísticas}



\subsection{Procura em Mmbientes Parcialmente Observáveis}



\subsection{Agentes de Procura Online e Ambientes Desconhecidos}



\newpage

\section{Procura Adversária}



\newpage

\section{Problemas de Satisfação de Restrições}



\newpage

\section{Planeamento Automático}



\newpage

\section{Aprendizagem por Reforço}



\end{document}