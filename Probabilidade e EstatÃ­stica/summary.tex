\documentclass[11pt, a4paper]{article}
\usepackage[margin=2cm]{geometry}
\usepackage{hyperref}
\usepackage{enumitem}
\usepackage{amsmath,amssymb}

\hypersetup{
    colorlinks,
    citecolor=black,
    filecolor=black,
    linkcolor=black,
    urlcolor=black
}

\setlength{\parindent}{0pt}

\begin{document}

\begin{titlepage}
    \begin{center}
        \vspace*{1cm}

        \textbf{\LARGE Probabilidade e Estatística}
        \vspace{0.5cm}

        \Large Resumo
        \vspace{1.5cm}

        \textbf{Rafael Rodrigues}
        \vfill
        LEIC \\
        Instituto Superior Técnico \\
        2023/2024
    \end{center}
\end{titlepage}

\tableofcontents

\newpage
\section{Conceitos básicos de probabilidade}


\subsection{Experiência aleatória. Espaço de resultados e acontecimentos}


\subsection{Noção de probabilidade. Probabilidade condicionada e lei da probabilidade total}

\subsubsection*{Probabilidade condicional}

\begin{equation*}
    P(A|B) = \frac{P(A \cap B)}{P(B)} \text{ , se } P(B) > 0
\end{equation*}


\subsection{Teorema de Bayes}


\subsection{Acontecimentos independentes}

\newpage
\section{Variáveis aleatórias discretas e contínuas}


\subsection{Definição de variável aleatória. Função de distribuição. Função de massa de probabilidade e função de densidade de probabilidade}


\subsection{Valor esperado, moda, variância e quantis}


\subsection{Distribuições de probabilidade mais utilizadas na modelação de dados: binomial, geométrica e de Poisson (discretas); uniforme, exponencial e normal (contínuas).}

\newpage
\section{Pares aleatórios}


\subsection{Distribuição conjunta, marginais e condicionais}


\subsection{Independência}


\subsection{Covariância e correlação}

\newpage
\section{Combinações lineares de variáveis aleatórias e teorema do limite central}


\subsection{Combinações lineares de variáveis aleatórias}


\subsection{Distribuição assintótica da soma e da média de variáveis aleatórias independentes e identicamente distribuídas}

\newpage
\section{Estimação pontual}


\subsection{Estatísticas e estimadores}


\subsection{Método da máxima verosimilhança}

\newpage
\section{Estimação intervalar}


\subsection{Intervalos de confiança para o valor esperado, variância conhecida (população normal ou com distribuição arbitrária)}


\subsection{Intervalos de confiança para o valor esperado, variância desconhecida (população normal ou com distribuição arbitrária)}


\subsection{Intervalo de confiança para a variância, valor esperado desconhecido (população normal)}


\subsection{Intervalo de confiança para uma probabilidade de sucesso (população de Bernoulli)}

\newpage
\section{Testes de hipóteses}

\subsection{Testes de hipóteses para o valor esperado, variância conhecida (população normal ou com distribuição arbitrária)}

\subsection{Testes de hipóteses para o valor esperado, variância desconhecida (população normal ou com distribuição arbitrária)}

\subsection{Testes de hipóteses para a variância valor esperado desconhecido (população normal)}

\subsection{Testes de hipóteses para uma probabilidade de sucesso (população de Bernoulli)}

\subsection{Teste de ajustamento do qui-quadrado de Pearson para hipótese nula simples}


\newpage
\section{Introdução à regressão linear simples}

\subsection{Modelo de regressão linear simples}

\subsection{Intervalos de confiança e testes de hipóteses para os parâmetros $\beta_0$, $\beta_1$ e $\beta_0 + \beta_1x_0$ do modelo de regressão linear simples}

\subsection{Coeficiente de determinação}


\end{document}