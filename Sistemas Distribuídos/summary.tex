\documentclass[12pt]{article}
\usepackage[paper=letterpaper,margin=2cm]{geometry}
\usepackage{enumitem}
\usepackage{amsmath}
\usepackage{amsfonts}
\usepackage{tabularray}
\usepackage{mathtools}

\setlength{\parindent}{0pt}


\begin{document}

\begin{titlepage}
    \begin{center}
        \vspace*{1cm}

        \textbf{\LARGE Sistemas Distribuídos}
        \vspace{0.5cm}

        \Large Resumo
        \vspace{1.5cm}

        \textbf{Rafael Rodrigues}
        \vfill
        LEIC \\
        Instituto Superior Técnico \\
        2022/2023
    \end{center}
\end{titlepage}

\tableofcontents

\newpage

\section{Introdução}

\textbf{Sistema Distribuído} - Sistema de componentes software ou hardware localizados em computadores ligados em rede que comunicam e coordenam as suas ações através de troca de mensagens.

\newpage

\section{Remote Procedure Call (RPC) }

Estrutura a programação distribuída com base na chamada (pelo clientes) de procedimentos que se executam remotamente (no servidor).

O cliente envia um pedido e o servidor envia uma resposta.

\subsection{Arquitetura cliente-servidor}

\begin{itemize}
    \item Servidores mantêm recursos e server pedidos de operações sobre esses recursos.
    \item Servidores podem ser clientes de outros servidores.
    \item Simples e permite distribuir sistemas centralizados muito diretamente.
    \item Limitado pela capacidade do servidor e pela rede que o liga aos clientes.
\end{itemize}

\newpage

\section{Sincronização de Relógios}

\begin{itemize}
    \item \textbf{Externa} - relógios dos processos são sincronizados através de uma \textbf{referência externa}
    \item \textbf{Interna} - relógios dos processos de um sistema \textbf{sincronizam-se entre si}
\end{itemize}

\subsection{Algoritmo de Cristian}

Os relógios dos clientes são sincronizados pelo relógio de um \textbf{servidor de tempo} (sincronização externa).

\begin{enumerate}
    \item Servidor $S$ lê o valor dos outros relógios. \\[2pt]
        $T_{S_i} = T_{env_i} + T_{rec_i}/2$ \\
        $delta_i = T_S - T_i$ \\
        $erro_i = \pm\ RTT_i/2$
    \item Indica a todos os participantes para ajustarem o seu relógio (incluindo o seu). \\[2pt]
        $ajuste_i = \overline{T} + delta_i $
\end{enumerate}

Diferença máxima = Soma dos dois maiores valores de erro

Precisão = $\pm \left(RTT/2 - min\right) $

\subsection{Algoritmo de Berkeley}

\begin{enumerate}
    \item É escolhido um líder através de um processo de eleição.
    \item O líder pergunta os tempos aos seus servidores.
    \item O líder calcula o tempo de cada máquina tendo em atenção o RTT.
    \item O líder calcula a média dos tempos, ignorando os outliers.
    \item Envia o valor (positivo ou negativo) que cada máquina deve ajustar.
\end{enumerate}

\subsection{Network Time Protocol (NTP)}

\newpage

\section{Relógios Lógicos}

Relacão happens-before/aconteceu-antes ($\rightarrow$)

\begin{enumerate}
    \item se $a$ e $b$ são eventos do mesmo processo, se $a$ ocorre antes de $b$, então $a \rightarrow b$
    \item se $a$ indica um evento envio de mensagem, e $b$ o evento da receção dessa mensagem, então $a \rightarrow b$
\end{enumerate}

\textbf{Transitividade}: se $a \rightarrow b$ e $b \rightarrow c$, então $a \rightarrow c$ \\
\textbf{Eventos concorrentes}: se nem $a \rightarrow b$, nem $b \rightarrow a$, então $a \parallel b$

\subsection{Lamport}

\begin{enumerate}
    \item se $a \rightarrow b$, então $C(a) < C(b)$
    \begin{itemize}[topsep=0pt]
        \item se os eventos ocorrerem no mesmo processo, e $a$ ocorre $antes$ de $b$, então $C(a) < C(b)$
        \item se $a$ for o evento envio de mensagem e $b$ a sua receção, então $C(a) < C(b)$
    \end{itemize}
    \item o valor de $C(e)\ nunca\ decresce$
    \begin{itemize}[topsep=0pt]
        \item As correções ao relógio devem ser feitas sempre por incrementos
    \end{itemize}
\end{enumerate}

\subsection{Vector clocks}

\begin{enumerate}
    \item se $C(a) < C(b)$, então $a \rightarrow b$
    \begin{itemize}[topsep=0pt]
        \item $V_a < V_b$ se pelo menos um elemento de $V_a$ for menor e nenhum for maior que $V_b$
    \end{itemize}
\end{enumerate}

\section{Registos}

\section{Espaços de tuplos}

\newpage

\section{Estados Globais}

\newpage

\section{Eleição de líder}

\subsection{Algoritmo baseado em anel}

\subsection{Algoritmo "bully"}

\section{Exclusão Mútua}

\subsection{Algoritmo de Ricart-Agrawala}

\subsection{Algoritmo de Maekawa}

\newpage

\section{Difusão em ordem total}

\newpage

\section{Replicação}

\subsection{Coerência forte}

\subsubsection{Primary-backup}

\subsubsection{Replicação de máquina de estados}

\subsubsection{Registo distribuído coerente}

\subsection{Coerência fraca}

\subsubsection{Gossip}

\subsubsection{Bayou}

\newpage

\section{Consenso}

\subsection{Floodset consensus}

\newpage

\section{Transações distribuídas}

Uma transação distribuída implica executar múltiplas suboperações em nós diferentes.
\begin{itemize}
    \item Caso tudo corra bem, todas as suboperações são confirmadas \textbf{(commit)}.
    \item Caso algo corra mal, todas as suboperações são anuladas \textbf{(abort)}.
\end{itemize}

Propriedades ACID:
\begin{itemize}
    \item Atomic - Atomicidade
    \item Consistent - Coerência
    \item Isolated - Isolamento
    \item Durable - Durabilidade
\end{itemize}

\subsection{2-phase commit (2PC)}

\begin{enumerate}
    \item O coordenador envia uma mensagem denominada "prepare" a todos os participantes.
    \item 
\end{enumerate}

\newpage

\section{Segurança}

\end{document}