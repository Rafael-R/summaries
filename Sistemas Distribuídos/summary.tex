\documentclass[12pt]{article}
\usepackage[paper=letterpaper,margin=2cm]{geometry}
\usepackage{enumitem}
\usepackage{amsmath}
\usepackage{amsfonts}
\usepackage{tabularray}
\usepackage{mathtools}

\setlength{\parindent}{0pt}


\begin{document}

\begin{titlepage}
    \begin{center}
        \vspace*{1cm}

        \textbf{\LARGE Sistemas Distribuídos}
        \vspace{0.5cm}

        \Large Resumo
        \vspace{1.5cm}

        \textbf{Rafael Rodrigues}
        \vfill
        LEIC \\
        Instituto Superior Técnico \\
        2022/2023
    \end{center}
\end{titlepage}

\tableofcontents

\newpage

\section{Introdução}

\textbf{Sistema Distribuído} - Sistema de componentes software ou hardware localizados em computadores ligados em rede, que comunicam e coordenam as suas ações através de troca de mensagens.

\newpage

\section{Remote Procedure Call (RPC) }

Estrutura a programação distribuída com base na chamada (pelo clientes) de procedimentos que se executam remotamente (no servidor).

O cliente envia um pedido e o servidor envia uma resposta.

\subsection{Arquitetura cliente-servidor}

\begin{itemize}
    \item Servidores mantêm recursos e server pedidos de operações sobre esses recursos.
    \item Servidores podem ser clientes de outros servidores.
    \item Simples e permite distribuir sistemas centralizados muito diretamente.
    \item Limitado pela capacidade do servidor e pela rede que o liga aos clientes.
\end{itemize}

\subsection{Interfaces de comunicação}

\subsubsection{UDP}

\begin{itemize}[topsep=0pt, itemsep=0pt]
    \item \textbf{Socket sem ligação}: 
     \begin{itemize}[topsep=0pt, itemsep=0pt]
        \item Normalmente serve mais de 2 interlocutores (1 para n).
        \item Normalmente não fiável (perdas, duplicação, reordenação).
     \end{itemize}
    \item \textbf{Socket Datagram}: canal sem ligação, bidirecional, não fiável, interface tipo mensagem.
    \item Maior eficiência.
    \item Menor overhead.
\end{itemize}

\subsubsection{TCP}

\begin{itemize}[topsep=0pt, itemsep=0pt]
    \item \textbf{Socket com ligação}: 
    \begin{itemize}[topsep=0pt, itemsep=0pt]
        \item Normalmente serve 2 interlocutores.
        \item Normalmente fiável, bidirecional e garante sequencialidade.
    \end{itemize}
    \item \textbf{Socket Stream}: canal com ligação, bidirecional, fiável, interface tipo sequência de octetos.
    \item Oferece um canal fiável sobre uma \underline{rede não fiável}.
    \item Para cada invocação remota precisamos de \underline{2 pares de mensagens adicionais}.
    \item Gestão de fluxo é \underline{redundante} para as invocações simples do nosso sistema.
    \item ACKs \underline{desnecessários}.
    \item É usado caso as mensagens sejam \underline{maiores que um datagrama UDP}.
\end{itemize}

\subsection{Serialização}

\newpage

\subsection{Semânticas de execução}

\textbf{Função idempotente} - Operação que, se executada repetidamente, produz o mesmo estado no servidor e resultado devolvido ao cliente do que se só executada 1 vez

\begin{itemize}
    \item \textbf{Talvez (maybe)} - Protocolo não pretende tolerar nenhuma falta pelo que nada faz para recuperar de uma situação de erro.
    \begin{itemize}[topsep=0pt, itemsep=0pt]
        \item O stub cliente não recebe uma resposta num prazo limite
        \item O timeout expira e a chamada retorna com erro
        \item Neste caso o cliente não sabe se o pedido foi executado ou não
    \end{itemize}
    \item \textbf{Pelo-menos-uma-vez (at-least-once)} - Para serviços com funções idempotentes.
    \begin{itemize}[topsep=0pt, itemsep=0pt]
        \item O stub cliente não recebe uma resposta num prazo limite
        \item O timeout expira e o stub cliente repete o pedido até obter uma resposta
        \item Se receber uma resposta o cliente tem a garantia que o pedido foi executado pelo menos uma vez
        \item Para evitar que o cliente fique permanentemente bloqueado em caso de falha do servidor existe um segundo timeout mais amplo
    \end{itemize}
    \item \textbf{No-máximo-uma-vez (at-most-once)} - Protocolo de controlo tem de: (1) identificar os pedidos para detetar repetições no servidor; (2) manter estado no servidor acerca dos pedidos em curso ou que já foram atendidos.
    \begin{itemize}[topsep=0pt, itemsep=0pt]
        \item O stub cliente não recebe uma resposta num prazo limite
        \item O timeout expira e o stub cliente repete o pedido
        \item O servidor não executa pedidos repetidos
        \item Se receber uma resposta, o cliente tem a garantia que o pedido foi executado no máximo uma vez
    \end{itemize}
    \item \textbf{Exatamente-uma-vez (exactly-once)} - Servidor e cliente com funcionamento transacional
    \begin{itemize}
        \item O stub cliente não recebe uma resposta num prazo limite
        \item O timeout expira e o stub cliente repete o pedido
        \item O servidor não executa pedidos repetidos
        \item Se o servidor falhar existe a garantia de fazer rollback ao
        estado do servidor de modo a não ficar com pedidos “a meio”
    \end{itemize}
\end{itemize}

\subsection{Serviços de Nomes}

\newpage

\subsection{Web Services}

\subsubsection{SOAP}

Desvantagens:
\begin{itemize}[topsep=4pt, itemsep=0pt]
    \item Obriga ao uso de XML
    \item Desempenho e escalabilidade limitados
    \begin{itemize}[topsep=0pt, itemsep=0pt]
        \item Representações em XML são longas, marshalling/unmarshalling é pesado
        \item Não permite caching de dados do lado do cliente
    \end{itemize}
    \item Relativamente difícil de implementar
\end{itemize}

\subsubsection{REST}

Vantagens:
\begin{itemize}[topsep=4pt, itemsep=0pt]
    \item Permite uso de alternativas ao XML, como o JSON (mais eficiente).
    \item Clientes que pretendam agir sobre um recurso, recebem cópia por inteiro dos dados
    \begin{itemize}[topsep=0pt, itemsep=0pt]
        \item Permite caching no cliente (melhor escalabilidade)
    \end{itemize}
\end{itemize}

\newpage

\subsection{gRPC}

É um mecanismo de RPC moderno, com estrutura semelhante aos RPC tradicionais, mas desenvolvido com foco em serviços cloud. 

Usado em comunicação para sistemas distribuídos com requisitos de baixa latência e elevada escalabilidade.

É um protocolo \textbf{eficiente}, \textbf{preciso} e \textbf{independente da linguagem} de programação.

\vspace{12pt}
Características do sistema gRPC:
\begin{itemize}[topsep=4pt, itemsep=0pt]
    \item O gRPC usa uma IDL (protobuf) para definir os tipos de dados e as operações.
    \item Disponibiliza ferramenta de geração de código a partir do IDL
    \begin{itemize}[topsep=0pt, itemsep=0pt]
        \item Trata da conversão de dados
        \item Gestão da invocação remota
    \end{itemize}
    \item Permite ter chamadas remotas síncronas e assíncronas
    \begin{itemize}[topsep=0pt, itemsep=0pt]
        \item \textbf{Chamadas síncronas} - esperam pela resposta
        \item \textbf{Chamadas assíncronas} - a resposta é verificada mais tarde
    \end{itemize}
\end{itemize}

\subsubsection{Protocol Buffers (protobuf)}

\begin{itemize}
    \item Os protocol buffers (protobuf) são uma IDL para descrever mensagens e operações
    \begin{itemize}
        \item Permite estender/modificar os esquemas, mantendo compatibilidade
        com versões anteriores
        \item O protobuf é um formato canónico
        \item A apresentação de dados tem uma estrutura explícita
    \end{itemize}
    \item O compilador protoc converte a IDL em código para uma grande
    variedade de linguagens de programação
    \item As camadas mais baixas do gRPC não dependem da IDL
    \begin{itemize}
        \item Em teoria é possível usar alternativas ao protobuf
    \end{itemize}
\end{itemize}

Formato \textbf{mais eficiente} que JSON e XML.\\
Permite uma boa integração com várias linguagens

\subsubsection{Stubs}

Objeto usado pelo cliente, que expõe a interface do serviço. A invocação de um método no stub vai serializar e traduzir os dados do pedido. Segue-se depois a invocação remota. É no stub que as restrições de interface e dos tipos de dados são verificadas.

\subsubsection{Run-time}

\subsubsection{Gestor de nomes}

\newpage

\section{Sincronização de Relógios}

\begin{itemize}
    \item \textbf{Externa} - relógios dos processos são sincronizados através de uma \textbf{referência externa}
    \item \textbf{Interna} - relógios dos processos de um sistema \textbf{sincronizam-se entre si}
\end{itemize}

\subsection{Algoritmo de Cristian}

Os relógios dos clientes são sincronizados pelo relógio de um \textbf{servidor de tempo} (sincronização externa).

\begin{enumerate}
    \item Servidor $S$ lê o valor dos outros relógios. \\[2pt]
        $T_{S_i} = T_{env_i} + T_{rec_i}/2$ \\
        $delta_i = T_S - T_i$ \\
        $erro_i = \pm\ RTT_i/2$
    \item Indica a todos os participantes para ajustarem o seu relógio (incluindo o seu). \\[2pt]
        $ajuste_i = \overline{T} + delta_i $
\end{enumerate}

Diferença máxima = Soma dos dois maiores valores de erro

Precisão = $\pm \left(RTT/2 - min\right) $

\subsection{Algoritmo de Berkeley}

\begin{enumerate}
    \item É escolhido um líder através de um processo de eleição.
    \item O líder pergunta os tempos aos seus servidores.
    \item O líder calcula o tempo de cada máquina tendo em atenção o RTT.
    \item O líder calcula a média dos tempos, ignorando os outliers.
    \item Envia o valor (positivo ou negativo) que cada máquina deve ajustar.
\end{enumerate}

\subsection{Network Time Protocol (NTP)}

\newpage

\section{Relógios Lógicos}

Relacão happens-before/aconteceu-antes ($\rightarrow$)

\begin{enumerate}
    \item se $a$ e $b$ são eventos do mesmo processo, se $a$ ocorre antes de $b$, então $a \rightarrow b$
    \item se $a$ indica um evento envio de mensagem, e $b$ o evento da receção dessa mensagem, então $a \rightarrow b$
\end{enumerate}

\textbf{Transitividade}: se $a \rightarrow b$ e $b \rightarrow c$, então $a \rightarrow c$ \\
\textbf{Eventos concorrentes}: se nem $a \rightarrow b$, nem $b \rightarrow a$, então $a \parallel b$

\subsection{Lamport}

\begin{enumerate}
    \item se $a \rightarrow b$, então $C(a) < C(b)$
    \begin{itemize}[topsep=0pt]
        \item se os eventos ocorrerem no mesmo processo, e $a$ ocorre $antes$ de $b$, então $C(a) < C(b)$
        \item se $a$ for o evento envio de mensagem e $b$ a sua receção, então $C(a) < C(b)$
    \end{itemize}
    \item o valor de $C(e)\ nunca\ decresce$
    \begin{itemize}[topsep=0pt]
        \item As correções ao relógio devem ser feitas sempre por incrementos
    \end{itemize}
\end{enumerate}

\subsection{Vector clocks}

\begin{enumerate}
    \item se $C(a) < C(b)$, então $a \rightarrow b$
    \begin{itemize}[topsep=0pt]
        \item $V_a < V_b$ se pelo menos um elemento de $V_a$ for menor e nenhum for maior que $V_b$
    \end{itemize}
\end{enumerate}

\section{Registos}

\subsection{Modêlos de Coerência}

\begin{itemize}
    \item Safe
    \begin{itemize}[topsep=0pt]
        \item Apenas um processo pode escrever e vários podem ler (1 escritor - múltiplos leitores)
        \item Se uma leitura não for concorrente com uma escrita, lê o último valor escrito.
        \item Se uma leitura for concorrente com uma escrita, pode retornar um valor arbitrário
    \end{itemize}
    \item Regular
    \begin{itemize}[topsep=0pt]
        \item Apenas um processo pode escrever e vários podem ler (1 escritor - múltiplos leitores)
        \item Se uma leitura não for concorrente com uma escrita, lê o último valor escrito.
        \item Se uma leitura for concorrente com uma escrita, retorna o valor anterior ou o valor que está a ser escrito
    \end{itemize}
    \item Atomic
    \begin{itemize}[topsep=0pt]
        \item O resultado da execução é equivalente ao resultado de uma execução em que todas as escritas e leituras ocorrem instantaneamente num ponto entre o início e o fim da operação.
    \end{itemize}
\end{itemize}

\subsection{Algoritmo ABD}



\newpage

\section{Espaços de tuplos}

Operações num espaço de tuplos:
\begin{itemize}[topsep=4pt, itemsep=0pt]
    \item Put: coloca um tuplo no espaço
    \item Read: lê um tuplo do espaço
    \item Take: lê e remove um tuplo do espaço
\end{itemize}

\vspace{10pt}
O Read e Take são \textbf{operações bloqueantes} (se não existir o
tuplo) e aceitam “wildcards”.\\

O Take, sendo bloqueante, permite sincronizar processos.\\

A exclusão mútua pode ser concretizada através de um tuplo $<token>$
\begin{itemize}[topsep=0pt, itemsep=0pt]
    \item um processo remove (Take) do espaço antes de aceder ao recurso
    \item e volta a colocar no espaço depois de aceder ao recurso\\
\end{itemize}

Abstração muito elegante pois permite partilha de memória e sincronização através de uma interface uniforme.\\

Num espaço de tuplos, podem existir várias cópias do mesmo tuplo.\\

O equivalente a uma escrita é feito através de um Take seguido de um Put.\\

O Take permite fazer operações que noutros tipos de memória partilhada mais “convencionais” requerem uma instrução do tipo compare-and-swap. \\

A operação Take requer ordem total.

\subsection{Implementação do Put}

\begin{enumerate}[itemsep=0pt]
    \item A réplica onde o pedido é feito faz multicast do Put para todas as outras réplicas e responde ao cliente
    \item Ao receber o pedido, as réplicas inserem o tuplo e enviam ACK
    \item O passo 1 é repetido até a réplica original receber o ACK de todas as outras réplicas
\end{enumerate}

\subsection{Implementação do Read}

\begin{enumerate}
    \item A réplica onde o pedido é feito faz multicast do Read para todas as outras réplicas
    \item Ao receber o pedido, as réplicas retornam o tuplo que faz “match”
    \item Ao receber a primeira resposta, a réplica retorna ao cliente, ignorando as respostas seguintes
    \item O passo 1 é repetido até se receber uma resposta
\end{enumerate}

\subsection{Implementação do Take}



\newpage

\section{Estados Globais}

Dado um evento $e_2$ pertencente ao corte, se $e_1 \rightarrow e_2$ então, para o corte ser coerente, $e_1$ tem de pertencer ao corte. Se $e_1 \rightarrow e_2$ e apenas $e_1$ pertencer ao corte, este continua a ser coerente.

\section{Eleição de líder}

\subsection{Algoritmo baseado em anel}



\subsection{Algoritmo "bully"}



No melhor cenário, o algoritmo de "Bully" consegue terminar a eleição trocando menos mensagens que o de anel.

\section{Exclusão Mútua}

\subsection{Algoritmo de Ricart-Agrawala}

\subsection{Algoritmo de Maekawa}

\newpage

\section{Difusão em ordem total}

\newpage

\section{Replicação}

\subsection{Coerência forte}

\subsubsection{Primary-backup (replicação passiva)}

\begin{enumerate}
    \item FE envia pedido ao primário (usando no-máximo-uma-vez)
    \item Primário ordena os pedidos por ordem de chegada
    \item Primário executa pedido e guarda resposta
    \item Primário envia aos secundários: (novo estado, resposta, id.pedido) e  espera por ack de todos os secundários
    \item Primário responde ao front-end, que retorna ao cliente
\end{enumerate}

\subsubsection{Replicação de máquina de estados (replicação ativa)}

\begin{enumerate}
    \item FE envia pedido (juntamente com um relógio lógico de Lamport) a \textbf{todas as réplicas usando difusão em ordem total} (mas sem esperar pelas réplicas em falha)
    \item O algoritmo de difusão entrega o pedido a todas (quando chegar a ver deste pedido na ordem total)
    \item Cada réplica executa o pedido
    \item Cada réplica responde ao cliente, que usa a primeira resposta
\end{enumerate}

Os pedidos necessitam de ser entregues por ordem total.

\subsubsection{Primary-backup vs RME}

\begin{itemize}
    \item Primário-secundário
    \begin{itemize}
        \item Suporta operações não deterministas (o líder decide o resultado)
        \item Se o líder produzir um valor errado, este valor é propagado para as réplicas
    \end{itemize}
    \item Replicação máquina de estados
    \begin{itemize}
        \item Se uma réplica produzir um valor errado não afeta as outras réplicas
        \item A operações necessitam de ser deterministas
    \end{itemize}
\end{itemize}

\subsection{Checkpoint recovery}

Técnica que para tolerar faltas guarda o estado da aplicação periodicamente, de forma a poder relançar a mesma sem ter de começar do início.

\subsection{Coerência fraca}

\subsubsection{Gossip}

Pedidos de leitura (query):
\begin{itemize}
    \item Réplica verifica se $pedido.prev \leq valueTS$
\end{itemize}

Pedidos de escrita (update):
\begin{itemize}
    \item responde e incrementa $pedido.prev_{[i]}$
\end{itemize}


\subsubsection{Bayou}

\newpage

\section{Consenso}

\begin{itemize}
    \item Conjunto de N processos
    \item Cada processo propõe um valor (input)
    \item Todos os processos decidem o mesmo valor (output)
    \begin{itemize}
        \item O valor decidido deve ser um dos valores propostos
        \item Pode ser qualquer um dos valores propostos
    \end{itemize}
\end{itemize}

\subsection{Floodset consensus}

\newpage

\section{Transações distribuídas}

Uma transação distribuída implica executar múltiplas suboperações em nós diferentes.
\begin{itemize}
    \item Caso tudo corra bem, todas as suboperações são confirmadas \textbf{(commit)}.
    \item Caso algo corra mal, todas as suboperações são anuladas \textbf{(abort)}.
\end{itemize}

Propriedades ACID:
\begin{itemize}
    \item Atomic - Atomicidade
    \item Consistent - Coerência
    \item Isolated - Isolamento
    \item Durable - Durabilidade
\end{itemize}

\subsection{2-phase commit (2PC)}

\begin{enumerate}
    \item O coordenador envia uma mensagem denominada "prepare" a todos os participantes.
    \item Se o participante pode confirmar a transação envia "ok" ao coordenador.
    \item Se o participante não puder confirmar a transação envia "not-ok" ao coordenador.
    \item Se o coordenador receber “ok” de todos os participantes decide:
    \begin{itemize}
        \item confirmar a transacção
        \item Acrescenta ao seu “log” uma entrada indicando que a transação foi confirmada
        \item e avisa os outros (que registam no seu log)
    \end{itemize}
    \item Se o coordenador receber “not-ok” de pelo menos um participante decide:
    \begin{itemize}
        \item abortar a transacção
        \item Acrescenta ao seu “log” uma entrada indicando que a transação foi
        abortada
        \item e avisa os outros (que registam no seu log)
    \end{itemize}
\end{enumerate}

Este algoritmo é bloqueante se o coordenador falha

\newpage

\section{Segurança}

\end{document}