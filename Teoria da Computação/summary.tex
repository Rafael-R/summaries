\documentclass[11pt]{article}
\usepackage[paper=letterpaper,margin=2cm]{geometry}
\usepackage{hyperref}
\usepackage{bookmark}
\usepackage{listings}
\usepackage{enumitem}
\usepackage{titlesec}
\usepackage{amsmath,amsfonts,amssymb}
\usepackage{tabularray}

\hypersetup{
    colorlinks,
    citecolor=black,
    filecolor=black,
    linkcolor=black,
    urlcolor=blue
}

\lstset{
    language=C,
    basicstyle=\small\ttfamily,
    %numbers=left,
    %numberstyle=\footnotesize,
    frame=single,
    tabsize=2,
    breaklines=true,
    xleftmargin=.35em,
    xrightmargin=.35em,
    %framexleftmargin=1.6em,
    extendedchars=true,
    literate={à}{{\`a}}1 {á}{{\'a}}1 {ã}{{\~a}}1 {ç}{{\c{c}}}1 {é}{{\'e}}1 {“}{{\“}}1 {”}{{\”}}1 {ê}{{\^e}}1 {õ}{{\~o}}1 {í}{{\'i}}1 {ú}{{\'u}}1 {é }{{\'e }}2
}

\setitemize{topsep=-4pt,itemsep=0pt}
\SetTblrInner{rowsep=2pt}

\setcounter{tocdepth}{4}
\setcounter{secnumdepth}{4}

\titleformat{\paragraph}
{\normalfont\normalsize\bfseries}{\theparagraph}{1em}{}
\titlespacing*{\paragraph}
{0pt}{3.25ex plus 1ex minus .2ex}{1.5ex plus .2ex}

\begin{document}

\begin{titlepage}
    \centering
    \vspace*{4cm}
    \textbf{\LARGE Sistemas Operativos} \\[0.5cm]
    \Large Resumo \\[2cm]
    \textbf{Rafael Rodrigues} \vfill
    LEIC \\ Instituto Superior Técnico \\ 2023/2024
\end{titlepage}

\pdfbookmark[section]{\contentsname}{toc}
\tableofcontents

\setlength\parskip{1em plus 0.1em minus 0.2em}
\setlength{\parindent}{0pt}

\newpage

\section{Preliminares}

\subsection{Conjuntos, funções e cardinalidade}

\subsection{Alfabetos e linguagens}

\textbf{Alfabeto -} conjunto finito e não-vazio (de símbolos), representado por $\Sigma$.

\textbf{Palavra -} sequência finita de elementos de um alfabeto $\Sigma$. Denotamos por $\Sigma^*$ o conjunto de todas as palavras desse alfabeto. 

\textbf{Linguagem -} de um alfabeto $\Sigma$ é um subconjunto de $\Sigma^*$. Denotamos por $\mathcal{L}^\Sigma$ o conjunto de todas as linguagens sobre $\Sigma$.

\section{Autómatos e Linguagens}

\subsection{Linguagens Regulares}

\subsubsection{Autómatos finitos deterministas (AFD)}

\subsubsection{Equivalência e minimização}

\subsubsection{Autómatos finitos não-deterministas (AFND)}

\subsubsection{Propriedades das linguagens regulares}

\paragraph{Expressões Regulares}

\paragraph{Lema da Bombagem}

\subsection{Linguagens Independentes do Contexto}

\subsubsection{Autómatos de pilha (AP)}

\subsubsection{Propriedades das linguagens independentes do contexto}

\section{Máquinas de Turing}

\section{Teoria da Computabilidade}

\section{Complexidade Computacional}

\end{document}